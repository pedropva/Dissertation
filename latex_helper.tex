% ---------------------------------------
% author: Alan Livio
% site: github.com/alanlivio/latex-helper-functions
% ---------------------------------------

% ----------------------------------------
% work like layout
% ----------------------------------------
% \usepackage{wordlike}
% \usepackage[big]{layaureo}

% ----------------------------------------
% ABNT like layout
% ----------------------------------------
% \usepackage{lmodern}
% \setlength{\parskip}{0.2cm}
% \setlength{\parindent}{1.3cm}
% \usepackage{geometry}
%  \geometry{
%  a4paper,
%  total={170mm,257mm},
%  left=3cm,
%  top=3cm,
%  right=3cm,
%  bottom=3cm
%  }

% ----------------------------------------
% SmallCaps font
% ----------------------------------------
% \usepackage{fontspec}
% \defaultfontfeatures{Mapping=tex-text}
% \setmainfont[
% SmallCapsFont = Fontin-SmallCaps.otf,
% BoldFont = Fontin-Bold.otf,
% ItalicFont = Fontin-Italic.otf
% ]{Fontin.otf}

% ----------------------------------------
% language
% ----------------------------------------
\usepackage[brazilian, american]{babel}
% \usepackage[portuguese]{babel}

% ----------------------------------------
% fonts
% ----------------------------------------
\usepackage[T1]{fontenc}
\usepackage[utf8]{inputenc}
\usepackage{textcomp}
\usepackage{amsmath,amssymb,amsfonts}

% ----------------------------------------
% tables
% ----------------------------------------
\usepackage{makecell}
\usepackage{tabularx}
\usepackage{multirow}
%\usepackage[nottoc]{tocbibind}

% ----------------------------------------
% table custom cells
% ----------------------------------------
\newcolumntype{L}[1]{>{\raggedright\arraybackslash}p{#1}}
\newcolumntype{C}[1]{>{\centering\arraybackslash}p{#1}}
\newcolumntype{R}[1]{>{\raggedleft\arraybackslash}p{#1}}

% ----------------------------------------
% URL and hyperref
% ----------------------------------------
%\usepackage[hyphens]{url}
%\usepackage{hyperref}
%\hypersetup{hidelinks}

% ----------------------------------------
% URL color
% ----------------------------------------
% \definecolor{linkcolour}{rgb}{0,0.2,0.6}
% \hypersetup{colorlinks,urlcolor=linkcolour, linkcolor=linkcolour}

% ----------------------------------------
% includegraphics
% ----------------------------------------
\usepackage{graphicx}

% ----------------------------------------
% includegraphics for larging image with subcaption
% ----------------------------------------
\usepackage{float}
\usepackage{newfloat}
\usepackage{subfloat}
\usepackage{subcaption}

% ----------------------------------------
% itemize
% ----------------------------------------
\usepackage{enumitem}
\SetEnumitemKey{mynosep}{noitemsep, nosep, topsep=0pt, partopsep=0pt, parsep=0pt, itemsep=0pt, leftmargin=*}
\def\labelitemi{$-$}

% ----------------------------------------
% color
% ----------------------------------------
\RequirePackage{color}
\usepackage{soulutf8}
\usepackage{xcolor}
\usepackage{multicol}

\colorlet{light-gray}{gray!20}
\usepackage{listings}
\DeclareFloatingEnvironment[fileext=lst,placement={!htbp},name=Listing]{listing}

\usepackage{color}
\definecolor{dkgreen}{rgb}{0,0.6,0}
\definecolor{gray}{rgb}{0.5,0.5,0.5}
\definecolor{mauve}{rgb}{0.58,0,0.82}
\definecolor{gray}{rgb}{0.4,0.4,0.4}
\definecolor{darkblue}{rgb}{0.0,0.0,0.6}
\definecolor{lightblue}{rgb}{0.0,0.0,0.9}
\definecolor{cyan}{rgb}{0.0,0.6,0.6}
\definecolor{darkred}{rgb}{0.6,0.0,0.0}

\renewcommand{\ttdefault}{pcr}
\lstset{
  basicstyle=\ttfamily\footnotesize,
  columns=fullflexible,
  showstringspaces=false,
  numbers=left,                   % where to put the line-numbers
  numberstyle=\tiny\color{gray},  % the style that is used for the line-numbers
  stepnumber=1,
  numbersep=5pt,                  % how far the line-numbers are from the code
  backgroundcolor=\color{white},      % choose the background color. You must add \usepackage{color}
  showspaces=false,               % show spaces adding particular underscores
  showstringspaces=false,         % underline spaces within strings
  showtabs=false,                 % show tabs within strings adding particular underscores
  frame=none,                   % adds a frame around the code
  rulecolor=\color{black},        % if not set, the frame-color may be changed on line-breaks within not-black text (e.g. commens (green here))
  tabsize=2,                      % sets default tabsize to 2 spaces
  captionpos=b,                   % sets the caption-position to bottom
  breaklines=true,                % sets automatic line breaking
  breakatwhitespace=true,        % sets if automatic breaks should only happen at whitespace
  title=\lstname,                   % show the filename of files included with \lstinputlisting;
                                  % also try caption instead of title  
  commentstyle=\color{gray}\upshape
}


\lstdefinelanguage{XML}
{
  morestring=[s][\color{black}]{"}{"},
  morestring=[s][\color{black}]{>}{<},
  morecomment=[s]{<?}{?>},
  morecomment=[s][\color{dkgreen}]{<!--}{-->},
  stringstyle=\color{black},
  identifierstyle=\color{darkgray}\bfseries,
  keywordstyle=\color{black},
  morekeywords={xmlns,xsi,noNamespaceSchemaLocation,type,id,x,y,source,target,version,tool,transRef,roleRef,objective,eventually}% list your attributes here
}


% ----------------------------------------
% fixme
% ----------------------------------------
\usepackage[nomargin,inline]{fixme}
\fxusetheme{color}
% \fxsetup{draft}
% \fxsetup{final}

% ----------------------------------------
% ref elements
% ----------------------------------------
\newcommand{\reffig}[1]{\figurename~\ref{#1}}
\newcommand{\reftab}[1]{Table~\ref{#1}}
\newcommand{\reflis}[1]{Listing~\ref{#1}}
\newcommand{\refappen}[1]{Appendix~\ref{#1}}
\newcommand{\refsec}[1]{Section~\ref{#1}}
\newcommand{\refsubsec}[1]{subsection~\ref{#1}}

% ----------------------------------------
% xml tags
% ----------------------------------------
\newcommand{\xml}[1]{\texttt{<#1>}}
\newcommand{\attr}[1]{\emph{#1}}

% ----------------------------------------
% compact section titles
% ----------------------------------------
% \usepackage[compact]{titlesec}
% \titlespacing{\section}{0pt}{*0}{*0}
% \titlespacing{\subsection}{0pt}{*0}{*0}
% \titlespacing{\subsubsection}{0pt}{*0}{*0}

% ----------------------------------------
% for
% ----------------------------------------
\usepackage{forloop}

% spacing control
% ----------------------------------------
% \setlength{\itemsep}{0pt}
% \setlength{\parskip}{0pt}
% \setlength{\parsep}{0pt}
% \setlength{\abovecaptionskip}{1pt}
% \setlength{\belowcaptionskip}{1pt}

% ----------------------------------------
% lipsum
% ----------------------------------------
\usepackage{lipsum}

% ----------------------------------------
% tree
% ----------------------------------------
\usepackage{tikz}
\usetikzlibrary{trees}

% % ----------------------------------------
% % minted listing inside tcolorbox
% % ----------------------------------------
% \usepackage{listings}
% \usepackage{tikz}
% \usepackage[many]{tcolorbox}
% \usepackage{minted}
% \tcbuselibrary{minted,breakable,skins,raster}
% \newtcblisting[]{mintedlistingtcblisting}[1][!ht]{
%   enhanced jigsaw,pad at break*=5mm,
%   fontsize=\footnotesize
%   colback=yellow!5,colframe=yellow!50!black,listing only,
%   listing engine=minted,
%   minted language=#1,
%   minted options={fontsize=\footnotesize,breaklines,autogobble,linenos,numbersep=3mm},
%   overlay={\begin{tcbclipinterior}\fill[red!20!blue!20!white] (frame.south west)
%   rectangle ([xshift=5mm]frame.north west);\end{tcbclipinterior}}
% }

% ----------------------------------------
% listing code inside tcolorbox
% ----------------------------------------
\usepackage{listings}
\usepackage[many]{tcolorbox}
\tcbuselibrary{listingsutf8}
\newtcblisting[]{mytcblisting}[1][!ht]{
  enhanced jigsaw,
  enforce breakable,
  pad at break*=1mm,
  colback=black!2,
  colframe=black!30,
  top=1mm, bottom=1mm, left=0mm, right=0mm,
  listing only,
  coltitle=black,
  % drop fuzzy shadow,
  listing options={
    % shape
    xleftmargin=0pt, % 0 is default
    xrightmargin=-4pt, % 0 is default
    % frame ---
    framesep=0pt,
    aboveskip=-6pt,
    belowskip=-6pt,
    captionpos=b, % sets the caption position
    % numbers ----
    % numberstyle=\footnotesize\ttfamily,
    numbers=left,
    % firstnumber=2
    numbersep=8pt,% how far the line-numbers are from the code
    stepnumber=1, % the step between two line-numbers.
    numberstyle=\tiny\ttfamily,
    % code ----
    language=#1,
    tabsize=2,
    showspaces=false, % show spaces with underscores
    showstringspaces=false, % underline spaces within strings
    showtabs=false, % show tabs using underscores
    breaklines=true, % sets automatic line breaking
    breakindent=8pt,
    basicstyle=\footnotesize\ttfamily,
    columns=fullflexible,
    keywordstyle=\color{black},
    morekeywords={
      <, >,
    },
  },
}

% ----------------------------------------
% draft watermark
% ----------------------------------------
\newcommand\enableDraftWatermark{
  \RequirePackage{draftwatermark}
  \SetWatermarkFontSize{0.5in}
  \SetWatermarkColor[gray]{.9}
  \SetWatermarkText{\parbox{12em}{\centering
      Unpublished working draft.\\
      Not for distribution.}}
}


% ----------------------------------------
% acm prepring
% ----------------------------------------
\newcommand\enableACMPreprint{
  \settopmatter{printacmref=false}
  \setcopyright{none}
  \renewcommand\footnotetextcopyrightpermission[1]{}
  \pagestyle{plain}
}

% ----------------------------------------
% blind review authors
% ----------------------------------------
\newcommand\addACMBlindReviewAuthors{
  \author{Removed for double-blind review}
  \affiliation{%
    \institution{Removed for double-blind review}}

  \author{Removed for double-blind review}
  \affiliation{%
    \institution{Removed for double-blind review}}

  \author{Removed for double-blind review}
  \affiliation{%
    \institution{Removed for double-blind review}}

  \renewcommand{\shortauthors}{Removed for double-blind review}
}

\newcommand\addIEEEBlindReviewAuthors{
  \author{
  \IEEEauthorblockN{Removed for double-blind review}
  \IEEEauthorblockA{Removed for double-blind review  \\ Removed for double-blind review}
  \and
  \IEEEauthorblockN{Removed for double-blind review}
  \IEEEauthorblockA{Removed for double-blind review  \\ Removed for double-blind review}
  \and
  \IEEEauthorblockN{Removed for double-blind review}
  \IEEEauthorblockA{Removed for double-blind review  \\ Removed for double-blind review}
  }
}
