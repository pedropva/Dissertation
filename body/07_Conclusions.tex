%%% -*- coding: utf-8 -*-
\newpage

\chapter{Conclusions}
\label{chap:conclusions}

In this work we propose a multi-modal approach to sensitive content detection in video. Our model uses both audio and visual features. It uses pre-trained convolutional neural networks that are relatively simple to understand, and  applies a late-fusion feature method, which is simpler than the early fusion approach since we use a single model to classify both features.

We intend to evaluate our models by testing on other known datasets. We expect to validate this simple approach while maintaining a similar performance to the existing methods.

It is important to note that our method is not focused on mobile platforms, therefore memory and disk space are not major constraints. 

It is worth mentioning that our results are not directly comparable to existing ones since our definitions of violence do not match the ones used in other works. 

\section{Currently published papers on this work}
\label{sec:contrib}

By the the time of this writing, two papers have already been published in  conferences: \cite{2019NSFWbaseline}, and \cite{2020PornDetectionSBIE}, and we have finished the construction of the large scale dataset for sensitive content detection, to be published soon.

\section{Future Work}
\label{sec:future}


% após o fim desse trabalho é claro que ainda muitos locais em nossa abordagem para melhoria, mas somente com uma extraçaõ de features generalista e baselines, já conseguimos alcançar abordagens estado da arte
%imagino que a partir daqui, o sentido do progresso do trabalho é diminuir e simpleificar a extração de features, ao mesmo tempo especializando, ou seja, treinando a extração de features para a prórpia tarefa
         
Test if a sequential model can outperform a non-sequential model in specific cases that demand long term memory, such as long videos with very small sensitive scenes.

\section{Acknowledgements}
\label{sec:acks}
This work was supported by the Artificial Intelligence challenge, created by Brazil's National Research Net (RNP) and Microsoft, in 2019.