%%% -*- coding: utf-8 -*-
\newpage

\chapter{Sensitive Content Dataset}
\label{chap:dataset}

In this Chapter, we present our \textit{dataset} and the metrics to evaluate models in this dataset.

\section{The Sensitive Content Dataset - 120k}\label{sec:dataset}
\pva{confirmar se vou usar 120k mesmo (aqui e no datasheet)}
% Falar do sampling inicial, da presentatividade dos sets
% falar do balanceamento e do sampling do conjunto de teste
% Falar das talbeas no texto
% colocar o resto das tabelas talvez no appendix
% estatisticas depois de dropar os maiores e os menores videos, sem balanceamento
% usamos o balanceamento dropando apesa porn videos pq os gore sao poucos
% iremos distribuir o dataset na versao balanceada e sem balanceamento
% Comparar nossas escolhas de corte de tempo com o do yt 
% Checked for duplicates by name and size and by fdupes, sem garantias de de não ter subvideos


% Before clipping video based on duration
% Mean:  0:06:18.929511  STD:  0:12:18.237458
% There are 1355 (1.05%) videos longer than 00:30:56
% There are 116 (0.09%) videos shorter than 00:00:05

% Videos:
% Total dataset: 127075
% Total size of videos (calculated individually): 3.5TiB
% Total duration of videos (calculated indiviadually): 11806:21:13
% Improper videos: 67424
% Total size of videos (calculated individually): 1.2TiB
% Total duration of videos (calculated indiviadually): 6953:27:41
% Proper videos: 59651
% Total size of videos (calculated individually): 2.2TiB
% Total duration of videos (calculated indiviadually): 4852:53:31
% Total size of features: '896.3GiB'

% Hot much detail do we go on about the dataset's subclasses? (subclasses por porn and yt, no subclasses on gore)

Our \textit{dataset} is structured into two main classes: videos containing ``safe'' content and videos containing sensitive content.
It is divided into 59,651 safe videos and 67,424 videos with sensitive content.

\begin{table}
\centering
\label{tab:general-stats}
\caption{General statistics of the two main classes of the dataset, tag coverage refers to main tag annotation existence (videos may also have sub tags but no main tag).}
\begin{tabular}{c|r|r} 
\multicolumn{1}{l|}{} & Sensitive  & Safe        \\ 
\hline
Video Count           & 67424      & 59651       \\ 
\hline
Total Duration        & 6953:27:41 & 4852:53:31  \\ 
\hline
Mean Duration         & 00:06:11   & 00:04:52    \\ 
\hline
STD Duration          & 00:04:12   & 00:03:26    \\ 
\hline
Max Duration          & 00:30:55   & 00:30:55    \\ 
\hline
Min Duration          & 00:00:05   & 00:00:05    \\ 
\hline
Total Size            & 1.2TiB     & 2.2TiB      \\ 
\hline
Mean Size             & 19.3MiB    & 39.0MiB     \\ 
\hline
STD Size              & 35.4MiB    & 42.3MiB     \\ 
\hline
Features Size         & 519.4GiB   & 376.8GiB    \\ 
\hline
Tag coverage          & 65392      & 51011       \\ 
\hline
Tag coverage (\%)     & 96,9862    & 85,5157     \\
\end{tabular}
\end{table}
% \section{SFW Videos}\label{sec:safe_videos}

For \textit{safe content}, we extracted 50,988 videos from Youtube8M\footnote{\url{https://research.google.com/youtube8m}}~\cite{abu2016youtube}. We chose this dataset because of the size of the dataset (8 million videos) and because of the wide variety of video classification challenges it supports.

We also included 8,663 videos collected from Youtube, hereby refered as cherry-picked, those videos were selected for the purpose of incremeting the amount of "hard" videos, which are videos that could possibly be misclassified as sensitive, such as MMA, breastfeeding, pool party, beach and other videos that have a higher amount of skin exposure. The amount of cherry-picked videos collected are listed by its query in Table \ref{tab:non-yt-count}. The collection was made by automated means, an script automatically searches and downloads all videos from the first 100 result pages.
% One of the objectives of the project is to evaluate the presence of inappropriate videos in educational video repositories. Therefore, in selecting your own videos within YouTube8M, we have chosen videos from \ "Jobs \& Education" and "News" top-categories.
% We made this choice because these videos generally have the "talking head" format. We believe that this format is common in an educational video repository such as video@RNP.
% \footnote{\url{https://video.rnp.br}}.

% We also included the Cholec80 \cite{twinanda2016endonet} dataset, it contains 80 videos of cholecystectomy surgeries performed by 13 surgeons. All videos from the Cholec80 dataset were labeled as safe, since videos of surgery are usual in an educational context.

% \section{NSFW Videos}\label{sec:nsfw_videos}

For \textit{sensitive content}, we collected pornography and violent videography (thereafter referred to as \textit{gore}) from websites. For the porn category, we collected 54,549 videos from XVideos\footnote{\url{https://info.xvideos.com/db}}. We chose this source because of the database size (7 million videos) and because of the amount and variety of annotations. In this database, each video has one main tag,\section{Analysis cases}\label{sec:experiments-discussion}

 totalling 70 main tags, and none or many subtags (user-created). To select the videos in this database, we distributed videos in these tags to maintain a proportion equal to the original XVideos database. In particular, to prevent lower-quantity tags from disappearing, we have defined a minimum of 10 pornographic videos for each tag. 

For the gore content, we used a web crawler to extract 2,356 gore videos from various websites dedicated to gore media, such as, BestGore\footnote{\url{https://www.bestgore.com/}} and GoreBrasil\footnote{\url{https://www.gorebrasil.com}}. 

%\pva{o set de teste de gore n tem gore de verdade, incluir?}
%For testing purposes we collected a separate group of 671 violent videos from youtube channel BustedLocals\footnote{\url{https://www.youtube.com/channel/UCpKimtjklYBgSKfSJrnXLRA}}
\begin{table}
\centering
\label{tab:granular-stats}
\caption{Granular statistics of the dataset: Videos collected from Youtube, pornographic videos, and gore videos.}
\begin{tabular}{c|r|r|r} 
\multicolumn{1}{l|}{} & \multicolumn{1}{c|}{Porn} & \multicolumn{1}{c|}{Gore} & \multicolumn{1}{c}{YouTube}  \\ 
\hline
Video Count           & 65068                     & 2356                      & 59651                         \\ 
\hline
Total Duration        & 6900:17:38                & 53:10:02                  & 4852:53:31                    \\ 
\hline
Mean Duration         & 00:06:21                  & 00:01:21                  & 00:04:52                      \\ 
\hline
STD Duration          & 00:04:10                  & 00:01:26                  & 00:03:26                      \\ 
\hline
Max Duration          & 00:30:55                  & 00:16:56                  & 00:30:55                      \\ 
\hline
Min Duration          & 00:00:05                  & 00:00:05                  & 00:00:05                      \\ 
\hline
Total Size            & 1.2TiB                    & 15.8GiB                   & 2.2TiB                        \\ 
\hline
Mean Size             & 19.8MiB                   & 6.9MiB                    & 39.0MiB                       \\ 
\hline
STD Size              & 35.9MiB                   & 13.9MiB                   & 42.3MiB                       \\ 
\hline
Features Size         & 515.3GiB                  & 4.1GiB                    & 376.8GiB                      \\ 
\hline
Tag coverage          & 63036                     & 2356                      & 51011                         \\ 
\hline
Tag coverage (\%)     & 96,8771                   & 100,0000                  & 85,5157                       \\
\end{tabular}
\end{table}

We hold out our dataset for testing and validating our approach: 10\% of the safe videos, then 10\% of gore videos, and sample a number of pornography to to match the amount of safe videos minus the amount of gore test samples, so that the test subset has a balanced amount of sensitive and safe videos, while keeping a valid amount of gore videos. 

\begin{table}
\centering
\label{tab:subset-stats}
\caption{Test subset statistics.}
\begin{tabular}{c|r|r|r} 
\multicolumn{1}{l|}{} & \multicolumn{1}{c|}{Safe} & \multicolumn{1}{c|}{Pornography} & \multicolumn{1}{c}{Gore}  \\ 
\hline
Video Count           & 5968                      & 5732                             & 236                        \\ 
\hline
Total Duration        & 443:35:36                 & 574:03:29                        & 05:22:20                   \\ 
\hline
Mean Duration         & 00:04:27                  & 00:06:00                         & 00:01:21                   \\ 
\hline
STD Duration          & 00:02:32                  & 00:03:44                         & 00:01:48                   \\ 
\hline
Max Duration          & 00:29:01                  & 00:30:29                         & 00:16:56                   \\ 
\hline
Min Duration          & 00:00:07                  & 00:00:05                         & 00:00:07                   \\ 
\hline
Total Size            & 204.8GiB                  & 80.2GiB                          & 1.5GiB                     \\ 
\hline
Mean Size             & 35.1MiB                   & 14.3MiB                          & 6.6MiB                     \\ 
\hline
STD Size              & 35.0MiB                   & 25.6MiB                          & 14.4MiB                    \\ 
\hline
Features Size         & 34.5GiB                   & 42.6GiB                          & 424.1MiB                   \\ 
\hline
Tag coverage          & 5679                      & 5695                             & 236                        \\ 
\hline
Tag coverage (\%)     & 951.575                   & 993.545                          & 100                        \\
\end{tabular}
\end{table}


As a complementary test dataset, we selected the NPDI 2k-pornography dataset~\cite{avila2013pooling}, it contains 1000 non-pornographic videos and 1000 pornographic videos. Those non-pornographic videos are comprised of ``hard'' and ``easy'' videos according to the likelihood of misclassification. Some examples of ``hard'' videos are those with high amounts of exposed skin, such as swimming and sumo fighting videos.

\begin{table}
\centering
\label{tab:2kdataset-stats}
\caption{NPDI 2k-pornography dataset statistics.}
\begin{tabular}{c|r|r} 
\multicolumn{1}{l|}{} & \multicolumn{1}{c|}{Porn} & \multicolumn{1}{c}{Non-Porn}  \\ 
\hline
Video Count           & 1000                      & 1000                           \\ 
\hline
Total Duration        & 100:30:32                 & 40:26:06                       \\ 
\hline
Mean Duration         & 00:06:01                  & 00:02:25                       \\ 
\hline
STD Duration          & 00:05:49                  & 00:02:17                       \\ 
\hline
Max Duration          & 00:33:40                  & 00:20:16                       \\ 
\hline
Min Duration          & 00:00:05                  & 00:00:02                       \\ 
\hline
Total Size            & 26.4GiB                   & 18.5GiB                        \\ 
\hline
Mean Size             & 27.0MiB                   & 18.9MiB                        \\ 
\hline
STD Size              & 31.1MiB                   & 21.9MiB                        \\ 
\hline
Features Size         & 7.6GiB                    & 3.1GiB                         \\ 
\hline
Tag coverage          & 1000                      & 1000                           \\ 
\hline
Tag coverage (\%)     & 100                       & 100                            \\
\end{tabular}
\end{table}

\section{Metrics}\label{sec:metrics}

% in binary classification, recall of the positive class is also known as “sensitivity”; recall of the negative class is “specificity”.
%https://en.wikipedia.org/wiki/Sensitivity_and_specificity
%https://en.wikipedia.org/wiki/Precision_and_recall

To evaluate each experiment and our approach, we will use Precision (P), Recall (R) and, most importantly, the weighted F2-score. In this section we present a contextualized explanation these metrics.

% FROM: de morereira et al 2019
%For assessing the performance of the pornography locators, we re-port thenormalized classification accuracyrate (ACC), and theF2measure(F2). Prior to explaining ACC, we need to definerecallandspecificityfrom the point of view of pornography localization.  Specificity, in turn, measures the capacity of alocator to correctly identify truly negative video seconds as so. A spe-cificity of only 50%, for example, means the system mislabels one inevery two seconds of non-pornographic content, wrongly identifying itas sensitive. In this vein, ACC is the mean of recall and specificity. Ahigher accuracy indicates a higher capability of separating porno-graphic video seconds from non-pornographic ones.F2measure, in turn, is a more complex metric that depends also onthe concept ofprecision. From the point of view of pornography loca-lization, precision expresses how many seconds are truly relevant (i.e.,pornographic), among all the ones that a locator identifies as such.Therefore, F2is the weighted harmonic mean of recall and precision,which gives twice more weight to recall than to precision, by means of a=β2parameter.Eq. (7)depicts the original Fβformula=+ ×××+Fβprecision  recallβ   precision  recall(1    ),β22(7)in which we use=β2. In doing so, F2lets us pay more attention to therecall of the solutions, rather than to their precision. This is usefulbecause, in the case of pornographyfiltering, false-negative answers areworse than the false-positive ones. It is less prejudicial to wrongly denythe access to non-pornographic content, than to wrongly disclose por-nographic content. Hence, we can consider that a solution with higherF2measure is better, because it cares more about how many porno-graphic video seconds are really beingfiltered out (recall), instead ofhow many“supposedly”positive seconds are indeed pornographic(precision)



In the context of sensitive content detection, \textit{true positives} are videos predicted as sensitive and are in fact, sensitive. Likewise, \textit{true negatives} are videos predicted as safe and are indeed safe. \textit{False positives} are videos predicted as sensitive, but were safe, the same goes for \textit{false negatives}, which are videos that were predicted as safe, but were predicted as sensitive.

% from the daniel moreira paper:
% thenormalized classification accuracyrate (ACC)ACC is the mean of recall and specificity. Ahigher accuracy indicates a higher capability of separating porno-graphic video seconds from non-pornographic ones

Precision (Equation~\ref{equation:precision}) measures how many videos predicted as sensitive (both true positives and false positives) are truly sensitive. The Recall (Equation~\ref{equation:recall})  measures how many truly positive videos were correctly identified.

\begin{multicols}{2}
  \begin{equation}
    \label{equation:precision}
    P = \frac{TP}{TP + FP}
  \end{equation}

  \begin{equation}
    \label{equation:recall}
    R = \frac{TP}{TP + FN}
  \end{equation}
  
\end{multicols}

Where $TP, TN, FP$, and $FN$ denote the examples that are true positives, true negatives, false-positives, and false negatives, respectively.

\begin{equation}
\label{equation:fbeta}
F_\beta = (1+\beta^2) \times \frac{P \times R}{(\beta^2 \times P) + R}
\end{equation}

The $F_\beta$-score, defined in Equation~\ref{equation:fbeta}, evaluates the classifier by the harmonic mean between Precision and Recall. To account for label imbalance, after calculating the F2-score metrics for each label, we find their average weighted by support (the number of true instances for each label). 

Most related works use either F1-score ($\beta=1$) or F2-score ($\beta=2$) metrics as their main evaluation metric. While the F1-score represents an balanced performance metric, the F2-score gives twice more weight to the recall than to precision, which means that the metric is more focused on the recall of a solution.

% in the case of pornographyfiltering, false-negative answers areworse than the false-positive ones. It is less prejudicial to wrongly denythe access to non-pornographic content, than to wrongly disclose por-nographic content. Hence, we can consider that a solution with higherF2measure is better, because it cares more about how many porno-graphic video seconds are really beingfiltered out (recall), instead ofhow many“supposedly”positive seconds are indeed pornographic(precision)
In this work, the F2-score represents an overall performance metric, while the precision and recall metrics can give insights on what the classifier model is doing better and what to improve. We chose the weighted F2-score as our main evaluation metric because when detecting sensitive content it is more important to predict a truly sensitive video than to predict a safe video as sensitive.
