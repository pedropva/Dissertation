%% -*- coding: utf-8; -*-

% DEFESA PEDRO - Principais questões levantadas pela banca

% 1) Prof Sandra

% - Não achou clara a motivação do pipeline
% - Sugeiriu categorização dos trabalhos relacionados e tabela comparativa
% - Motivar melhor o PCA e extrair a variância
% - Melhorar a explicação da 1a abordagem
% - Usar como base o artigo "Datasheets for Datasets" para documentar como foi montado o dataset e como ele deve ser usado
% - Fazer análises sobre a importância do áudio e do vídeo e ter conclusões sobre o uso em conjunto


% 2) Alberto

% - Colocar os agradecimentos microsoft e RNP explicando
% - não são 6 experimentos e sim análises


% 3) Julio

% - Extração de features para áudio: por que usar CNNs? 
% - Justificar melhor o PCA
% - Tabela comparativa dos trabalhos relacionados
% - un unicco trabalho de redes recorrents. Testar GRU (?)
% - COmo vai fazer a análise dos erros? 
% - Explicabilidade (?)


\documentclass[phd,american]{ThesisPUC}

%---------- Math ----------%
\usepackage{amsmath}
\usepackage{amssymb}
\usepackage{mathtools}
\usepackage{bbm}
%\usepackage{ulem}
%---------- Floting ----------%
\usepackage{float}
% ---------- References ----------%
%\usepackage[sort&compress,round,comma,numbers]{natbib}
%\usepackage{natbib}
% ---------------------------------------
% author: Alan Livio
% site: github.com/alanlivio/latex-helper-functions
% ---------------------------------------

% ----------------------------------------
% work like layout
% ----------------------------------------
% \usepackage{wordlike}
% \usepackage[big]{layaureo}

% ----------------------------------------
% ABNT like layout
% ----------------------------------------
% \usepackage{lmodern}
% \setlength{\parskip}{0.2cm}
% \setlength{\parindent}{1.3cm}
% \usepackage{geometry}
%  \geometry{
%  a4paper,
%  total={170mm,257mm},
%  left=3cm,
%  top=3cm,
%  right=3cm,
%  bottom=3cm
%  }

% ----------------------------------------
% SmallCaps font
% ----------------------------------------
% \usepackage{fontspec}
% \defaultfontfeatures{Mapping=tex-text}
% \setmainfont[
% SmallCapsFont = Fontin-SmallCaps.otf,
% BoldFont = Fontin-Bold.otf,
% ItalicFont = Fontin-Italic.otf
% ]{Fontin.otf}

% ----------------------------------------
% language
% ----------------------------------------
\usepackage[brazilian, american]{babel}
% \usepackage[portuguese]{babel}

% ----------------------------------------
% fonts
% ----------------------------------------
\usepackage[T1]{fontenc}
\usepackage[utf8]{inputenc}
\usepackage{textcomp}
\usepackage{amsmath,amssymb,amsfonts}

% ----------------------------------------
% tables
% ----------------------------------------
\usepackage{makecell}
\usepackage{tabularx}
\usepackage{multirow}
%\usepackage[nottoc]{tocbibind}

% ----------------------------------------
% table custom cells
% ----------------------------------------
\newcolumntype{L}[1]{>{\raggedright\arraybackslash}p{#1}}
\newcolumntype{C}[1]{>{\centering\arraybackslash}p{#1}}
\newcolumntype{R}[1]{>{\raggedleft\arraybackslash}p{#1}}

% ----------------------------------------
% URL and hyperref
% ----------------------------------------
%\usepackage[hyphens]{url}
%\usepackage{hyperref}
%\hypersetup{hidelinks}

% ----------------------------------------
% URL color
% ----------------------------------------
% \definecolor{linkcolour}{rgb}{0,0.2,0.6}
% \hypersetup{colorlinks,urlcolor=linkcolour, linkcolor=linkcolour}

% ----------------------------------------
% includegraphics
% ----------------------------------------
\usepackage{graphicx}

% ----------------------------------------
% includegraphics for larging image with subcaption
% ----------------------------------------
\usepackage{float}
\usepackage{newfloat}
\usepackage{subfloat}
\usepackage{subcaption}

% ----------------------------------------
% itemize
% ----------------------------------------
\usepackage{enumitem}
\SetEnumitemKey{mynosep}{noitemsep, nosep, topsep=0pt, partopsep=0pt, parsep=0pt, itemsep=0pt, leftmargin=*}
\def\labelitemi{$-$}

% ----------------------------------------
% color
% ----------------------------------------
\RequirePackage{color}
\usepackage{soulutf8}
\usepackage{xcolor}
\usepackage{multicol}

\colorlet{light-gray}{gray!20}
\usepackage{listings}
\DeclareFloatingEnvironment[fileext=lst,placement={!htbp},name=Listing]{listing}

\usepackage{color}
\definecolor{dkgreen}{rgb}{0,0.6,0}
\definecolor{gray}{rgb}{0.5,0.5,0.5}
\definecolor{mauve}{rgb}{0.58,0,0.82}
\definecolor{gray}{rgb}{0.4,0.4,0.4}
\definecolor{darkblue}{rgb}{0.0,0.0,0.6}
\definecolor{lightblue}{rgb}{0.0,0.0,0.9}
\definecolor{cyan}{rgb}{0.0,0.6,0.6}
\definecolor{darkred}{rgb}{0.6,0.0,0.0}

\renewcommand{\ttdefault}{pcr}
\lstset{
  basicstyle=\ttfamily\footnotesize,
  columns=fullflexible,
  showstringspaces=false,
  numbers=left,                   % where to put the line-numbers
  numberstyle=\tiny\color{gray},  % the style that is used for the line-numbers
  stepnumber=1,
  numbersep=5pt,                  % how far the line-numbers are from the code
  backgroundcolor=\color{white},      % choose the background color. You must add \usepackage{color}
  showspaces=false,               % show spaces adding particular underscores
  showstringspaces=false,         % underline spaces within strings
  showtabs=false,                 % show tabs within strings adding particular underscores
  frame=none,                   % adds a frame around the code
  rulecolor=\color{black},        % if not set, the frame-color may be changed on line-breaks within not-black text (e.g. commens (green here))
  tabsize=2,                      % sets default tabsize to 2 spaces
  captionpos=b,                   % sets the caption-position to bottom
  breaklines=true,                % sets automatic line breaking
  breakatwhitespace=true,        % sets if automatic breaks should only happen at whitespace
  title=\lstname,                   % show the filename of files included with \lstinputlisting;
                                  % also try caption instead of title  
  commentstyle=\color{gray}\upshape
}


\lstdefinelanguage{XML}
{
  morestring=[s][\color{black}]{"}{"},
  morestring=[s][\color{black}]{>}{<},
  morecomment=[s]{<?}{?>},
  morecomment=[s][\color{dkgreen}]{<!--}{-->},
  stringstyle=\color{black},
  identifierstyle=\color{darkgray}\bfseries,
  keywordstyle=\color{black},
  morekeywords={xmlns,xsi,noNamespaceSchemaLocation,type,id,x,y,source,target,version,tool,transRef,roleRef,objective,eventually}% list your attributes here
}


% ----------------------------------------
% fixme
% ----------------------------------------
\usepackage[nomargin,inline]{fixme}
\fxusetheme{color}
% \fxsetup{draft}
% \fxsetup{final}

% ----------------------------------------
% ref elements
% ----------------------------------------
\newcommand{\reffig}[1]{\figurename~\ref{#1}}
\newcommand{\reftab}[1]{Table~\ref{#1}}
\newcommand{\reflis}[1]{Listing~\ref{#1}}
\newcommand{\refappen}[1]{Appendix~\ref{#1}}
\newcommand{\refsec}[1]{Section~\ref{#1}}
\newcommand{\refsubsec}[1]{subsection~\ref{#1}}

% ----------------------------------------
% xml tags
% ----------------------------------------
\newcommand{\xml}[1]{\texttt{<#1>}}
\newcommand{\attr}[1]{\emph{#1}}

% ----------------------------------------
% compact section titles
% ----------------------------------------
% \usepackage[compact]{titlesec}
% \titlespacing{\section}{0pt}{*0}{*0}
% \titlespacing{\subsection}{0pt}{*0}{*0}
% \titlespacing{\subsubsection}{0pt}{*0}{*0}

% ----------------------------------------
% for
% ----------------------------------------
\usepackage{forloop}

% spacing control
% ----------------------------------------
% \setlength{\itemsep}{0pt}
% \setlength{\parskip}{0pt}
% \setlength{\parsep}{0pt}
% \setlength{\abovecaptionskip}{1pt}
% \setlength{\belowcaptionskip}{1pt}

% ----------------------------------------
% lipsum
% ----------------------------------------
\usepackage{lipsum}

% ----------------------------------------
% tree
% ----------------------------------------
\usepackage{tikz}
\usetikzlibrary{trees}

% % ----------------------------------------
% % minted listing inside tcolorbox
% % ----------------------------------------
% \usepackage{listings}
% \usepackage{tikz}
% \usepackage[many]{tcolorbox}
% \usepackage{minted}
% \tcbuselibrary{minted,breakable,skins,raster}
% \newtcblisting[]{mintedlistingtcblisting}[1][!ht]{
%   enhanced jigsaw,pad at break*=5mm,
%   fontsize=\footnotesize
%   colback=yellow!5,colframe=yellow!50!black,listing only,
%   listing engine=minted,
%   minted language=#1,
%   minted options={fontsize=\footnotesize,breaklines,autogobble,linenos,numbersep=3mm},
%   overlay={\begin{tcbclipinterior}\fill[red!20!blue!20!white] (frame.south west)
%   rectangle ([xshift=5mm]frame.north west);\end{tcbclipinterior}}
% }

% ----------------------------------------
% listing code inside tcolorbox
% ----------------------------------------
\usepackage{listings}
\usepackage[many]{tcolorbox}
\tcbuselibrary{listingsutf8}
\newtcblisting[]{mytcblisting}[1][!ht]{
  enhanced jigsaw,
  enforce breakable,
  pad at break*=1mm,
  colback=black!2,
  colframe=black!30,
  top=1mm, bottom=1mm, left=0mm, right=0mm,
  listing only,
  coltitle=black,
  % drop fuzzy shadow,
  listing options={
    % shape
    xleftmargin=0pt, % 0 is default
    xrightmargin=-4pt, % 0 is default
    % frame ---
    framesep=0pt,
    aboveskip=-6pt,
    belowskip=-6pt,
    captionpos=b, % sets the caption position
    % numbers ----
    % numberstyle=\footnotesize\ttfamily,
    numbers=left,
    % firstnumber=2
    numbersep=8pt,% how far the line-numbers are from the code
    stepnumber=1, % the step between two line-numbers.
    numberstyle=\tiny\ttfamily,
    % code ----
    language=#1,
    tabsize=2,
    showspaces=false, % show spaces with underscores
    showstringspaces=false, % underline spaces within strings
    showtabs=false, % show tabs using underscores
    breaklines=true, % sets automatic line breaking
    breakindent=8pt,
    basicstyle=\footnotesize\ttfamily,
    columns=fullflexible,
    keywordstyle=\color{black},
    morekeywords={
      <, >,
    },
  },
}

% ----------------------------------------
% draft watermark
% ----------------------------------------
\newcommand\enableDraftWatermark{
  \RequirePackage{draftwatermark}
  \SetWatermarkFontSize{0.5in}
  \SetWatermarkColor[gray]{.9}
  \SetWatermarkText{\parbox{12em}{\centering
      Unpublished working draft.\\
      Not for distribution.}}
}


% ----------------------------------------
% acm prepring
% ----------------------------------------
\newcommand\enableACMPreprint{
  \settopmatter{printacmref=false}
  \setcopyright{none}
  \renewcommand\footnotetextcopyrightpermission[1]{}
  \pagestyle{plain}
}

% ----------------------------------------
% blind review authors
% ----------------------------------------
\newcommand\addACMBlindReviewAuthors{
  \author{Removed for double-blind review}
  \affiliation{%
    \institution{Removed for double-blind review}}

  \author{Removed for double-blind review}
  \affiliation{%
    \institution{Removed for double-blind review}}

  \author{Removed for double-blind review}
  \affiliation{%
    \institution{Removed for double-blind review}}

  \renewcommand{\shortauthors}{Removed for double-blind review}
}

\newcommand\addIEEEBlindReviewAuthors{
  \author{
  \IEEEauthorblockN{Removed for double-blind review}
  \IEEEauthorblockA{Removed for double-blind review  \\ Removed for double-blind review}
  \and
  \IEEEauthorblockN{Removed for double-blind review}
  \IEEEauthorblockA{Removed for double-blind review  \\ Removed for double-blind review}
  \and
  \IEEEauthorblockN{Removed for double-blind review}
  \IEEEauthorblockA{Removed for double-blind review  \\ Removed for double-blind review}
  }
}

%---------- Algorithm ----------%
\input{conf_alg}
%---------- Tables ----------%
\usepackage{multicol}
\usepackage{booktabs}
\usepackage{colortbl}
\usepackage{tabularx}
%---------- Images ----------%
%\usepackage{subfigure}
%---------- TiKz ----------%
\usepackage{threeparttable}
\usepackage{standalone}
\usepackage{xcolor}
\usepackage{tikz}
\usepackage{psfrag,epsf}
\usepackage{multirow}
\usepackage{listings}
\usepackage{soul}
%----------citation------------%
%\usepackage{apacite}

%-----------color-----------------%
\definecolor{dkblue}{rgb}{0,0,.6}
\definecolor{turquoise}{rgb}{0.25,0.87,0.82}
\definecolor{dkgreen}{rgb}{0,0.6,0}
\definecolor{dkred}{rgb}{0.7,0,0}
\definecolor{indigo}{rgb}{0.294, 0, 0.51}
\definecolor{cyan}{rgb}{0, 0.70, 0.70}
\definecolor{gray}{rgb}{0.5,0.5,0.5}
\definecolor{mauve}{rgb}{0.58,0,0.82}
\definecolor{black}{rgb}{0,0,0}

\newcommand{\pva}[1]{\color{red}\textbf{TODO: }#1\color{black}}
%\pagestyle{plain}


%Para o código em EPL
\lstdefinestyle{EPLStyle}{
  numbers=left,
  numberstyle=\footnotesize\ttfamily,
  language=SQL,
  frame=tblr,
  aboveskip=0mm,
  belowskip=3mm,
  showstringspaces=false,
  columns=flexible,
  basicstyle={\small\ttfamily},
  numberstyle=\tiny\color{gray},
  keywordstyle=\color{blue},
  commentstyle=\color{gray},
  stringstyle=\color{dkgreen},
  breaklines=true,
  breakatwhitespace=true,
  tabsize=3,
  %Adiciona as keywords específicas da EPL Esper
  morekeywords={after, at, context, current_timestamp,  define, distinct, every, first, grouping, grouping_id, hour, hours, initiated, inner, instanceof, irstream, is, istream, last, match_recognize,  measures, min, minute, minutes, microsecond, microseconds, millisecond, milliseconds, msec, new, offset, output, partition, pattern, rstream, sec, second, seconds, sets, some, snapshot, sql, start, terminated, then, until, usec, using, variable, weekday, when, while, window, schema},
%
  frameround=tttt,
}

\graphicspath{{images/}}

% ---------- Cover ----------%
% Adjust the advisor's title according to gender(Prof. or Prof$^{\text{a}}$.)
\author{Pedro Vinicius Almeida de Freitas}
\authorR{Freitas, Pedro Vinicius Almeida de}
\advisor{Sérgio Colcher}{Prof.}
\advisorR{Colcher, Sérgio}

% This thesis will use colored figures, this goes in the catalographic sheet
\usecolour{true}
\title{Detecção de Conteúdo Sensível em Video com Deep Learning}

\titleuk{Sensitive Content Detection in Video with Deep Learning}

\day{16$^{th}$}
\month{Febuary}
\myyear{2022}

% CDD is the registry number of the area, given by the library. Our area (informatics) is 004.
\city{Rio de Janeiro}
\CDD{004}
\department{Informática}
\program{Informática}
\school{Centro Técnico Científico}
\university{Pontifícia Universidade Católica do Rio de Janeiro}
\uni{PUC-Rio}

%---------- Jury ----------%

% Internal jury members are declared with \jurymember{name}{title}{department}{university}
% external jury members are declared with \extjurymember{name}{title}{university}
\jury{
% TODO
\pva{PREENCHER BANCA}
%   \jurymember{Alberto Barbosa Raposo}{Prof.}{Departamento de Informática}{PUC-Rio}
%   \extjurymember{}{Dr.}{Disney Research}
}

%---------- Front letters ----------%
\resume
{
Bachelor's degree in Computer Science at Federal University of Maranhão (UFMA) in 2019.
}

\acknowledgment
{
\noindent
Thanks
% Thanks to my advisor Prof. Sérgio Colcher for his guidance and support in this journey.
% Thanks to my family for the endless support.
% Thanks to my friends from TeleMídia Lab, for their friendship and support. To all colleagues, faculty and staff of the PUC Rio Department of Informatics for the fellowship, learning and support.
% This study was financed in part by the Coordenação de Aperfeiçoamento de Pessoal de Nível Superior - Brasil (CAPES) - Finance Code 001.
}
% Abstract and Keywords

% Workaround for keywords. The keywords in the catalographic sheet must be separated by dots, while the ones shown in the abstract must be separated by semi-colons.
% Thats why we have two commands for each language: \keywords declares the keywords for the catalographic sheet, while \keywordsabstract declares the ones for the abstract.
\keywords
{
  \key{Conteúdo Sensível;}
  \key{Detecção de Conteúdo Sensível;}
  \key{Classificação Multimodal de Videos;}
  \key{Deep Learning.}
}

\keywordsabstract
{
  \key{Conteúdo Sensível;}
  \key{Detecção de Conteúdo Sensível;}
  \key{Classificação Multimodal de Videos;}
  \key{Deep Learning.}
}

\keywordsuk
{
  \key{Sensitive Content;}
  \key{Sensitive Video Dataset;}
  \key{Multimodal Video Classification;}
  \key{Deep Learning.}
}

\keywordsabstractuk
{
  \key{Sensitive Content;}
  \key{Sensitive Video Dataset;}
  \key{Multimodal Video Classification;}
  \key{Deep Learning.}
}

\abstract{
Grandes quantidades de vídeo são carregadas em plataformas de hospedagem de vídeo a cada minuto. Esse volume de dados apresenta um desafio no controle do tipo de conteúdo enviado para esses serviços de hospedagem de vídeo, pois essas plataformas são responsáveis por qualquer mídia sensível enviada por seus usuários.
Nesta dissertação, definimos conteúdo sensível como sexo, violencia fisica extrema, gore ou cenas potencialmente pertubadoras ao espectador. Apresentamos um conjunto de dados de vídeo sensível para classificação binária de vídeo (se há conteúdo sensível no vídeo ou não), contendo 127 mil vídeos anotados, cada um com seus embeddings visuais e de áudio extraídos. Também treinamos e avaliamos quatro modelos baseline para a tarefa de detecção de conteúdo sensível em vídeo. O modelo com melhor desempenho obteve 99\% de F2-Score ponderado no nosso subconjunto de testes e 88,83\% no conjunto de dados NPDI pornography-2k.
}

\abstractuk{
Massive amounts of video are uploaded on video-hosting platforms every minute. This volume of data presents a challenge in controlling the type of content uploaded to these video hosting services, for those platforms are responsible for any sensitive media uploaded by their users.
There has been an abundance of research on methods for developing automatic detection of sensitive content. In this dissertation, we define sensitive content as sex, extreme physical violence, gore or any scenes potentially disturbing to the viewer. We present a sensitive video dataset for binary video classification (whether there is sensitive content in the video or not), containing 127 thousand tagged videos, Each with their extracted audio and visual embeddings. We also trained and evaluated four baseline models on the sensitive content detection in video task. The best performing model archieved 99\% weighed F2-Score on our test subset and 88.83\% on the NPDI pornography-2k dataset.
}


% WARNING
% The epigraph, if present, must come before the first chapter, always.
% There is a list of abreviations (abrevs.tex) which is included automatically in the ThesisPUC.cls, and is optional, comment the %% -*- coding: utf-8; -*-

\begin{thenotations}
\renewcommand{\arraystretch}{1.5}
  \noindent
  \begin{tabular}{ll}

DL -- Deep Learning\\
CNN -- Convolutional Neural Network\\
RNN -- Recurrent Neural Network\\
LSTM -- Long Short Term Memory\\
AUC -- Area Under The Curve\\
ROC -- Receiver Operating Characteristics\\
NPDI -- Núcleo de Processamento Digital de Imagens\\
MAP -- Mean Average Precision\\
GRU -- Gated Recurrent Units\\
SVM -- Support Vector Machines\\
ANN -- Artificial Neural Network\\
MLP -- Multi Layer Perceptron\\
ILSVRC -- ImageNet Large Scale Visual Recognition Challenge\\
ReLU --  Rectified Linear Uni\\
RGB -- Red, Green and Blue\\
FC -- Fully Connected neural network\\
MMA -- Mixed Martial Arts\\
STD -- Standard Deviation\\
NSFW -- Not Safe For Work\\
KNN -- K-Nearest Neightbors\\
ANOVA -- Variance Analisys\\
RNP -- Brazil's National Research Net

\end{tabular}

\end{thenotations} line if you do not wish to included it.
% Rationale: In the original template, the list of abreviations came before the epigraph, which caused problems with the university library, thus I've included it in the cls file.
% TODO: Declare a boolean option in the ThesisPUC.cls class in order to selectively include the abreviations list.

%%%%%%%%%%%%%%%%%%%%%%%%%%%%%%%%%%%%%%%%%%%%%%%%%%%%%%
\begin{document}

% DEFESA PEDRO - Principais questões levantadas pela banca

% 1) Prof Sandra

% - Não achou clara a motivação do pipeline
% - Sugeiriu categorização dos trabalhos relacionados e tabela comparativa
% - Motivar melhor o PCA e extrair a variância
% - Melhorar a explicação da 1a abordagem
% - Usar como base o artigo "Datasheets for Datasets" para documentar como foi montado o dataset e como ele deve ser usado
% - Fazer análises sobre a importância do áudio e do vídeo e ter conclusões sobre o uso em conjunto


% 2) Alberto

% - Colocar os agradecimentos microsoft e RNP explicando
% - não são 6 experimentos e sim análises


% 3) Julio

% - Extração de features para áudio: por que usar CNNs? 
% - Justificar melhor o PCA
% - Tabela comparativa dos trabalhos relacionados
% - un unicco trabalho de redes recorrents. Testar GRU (?)
% - COmo vai fazer a análise dos erros? 
% - Explicabilidade (?)

\section{Introduction}
\label{sec:introduction}

% INTRO SLR
The amount of multimedia content on the internet is increasing each year.
More than 300 hours of video are uploaded to YouTube every minute.\footnote{\url{https://biographon.com/youtube-stats}}
In this context, studies have shown that about 56\% of children between 10 and 13 years old have a smartphone \cite{remosoftware,chollet2017xception}, and 8 out of 10 teenagers have had a friend who shared some sensitive media through social networks such as Facebook, Twitter, and Whatsapp.\footnote{https://www.netnanny.com/the-importance-of-parental-control/}


% In Brazil, the ``Cicarely case'' was an example that forced youtube to be blocked.\footnote{\url{http://g1.globo.com/Noticias/Tecnologia/0,,AA1412609-6174-363,00.html}}
% In our research, we are interested in helping to avoid scenarios where pornography can be uploaded to education channels, which might expose students, sometimes underage, to this kind of content.\footnote{\url{https://g1.globo.com/sp/sao-paulo/noticia/2020/06/19/professor-de-etec-na-zona-norte-de-sp-e-afastado-apos-se-masturbar-durante-aula-virtual.ghtml}}. 
%This scenery presents challenges on controlling which kind of contents are uploaded to this storage and distribution services, while dealing with great amounts of videos.

This huge amount of data sharing pattern presents a challenge to the control of the type of content that is loaded to these video repositories. By allowing the upload of sensitive content from malicious users, content providers become exposed to legal issues. This is also a problem for users in those platforms, as they might get exposed to this content without a warning.

Methods based on \textit{Deep Learning} (DL) became the \textit{state-of-the-art} in various segments related to automatic video analysis. More specifically, 
Convolutional Neural Networks (CNN) architectures, or ConvNets, have become the primary method used for audio-visual pattern recognition.

The term \emph{Sensitive content} is often used as a reference to any media that contains content such as nudity, intense sexuality, violence, gore, and any other potentially disturbing or offensive subject.
On the other hand, a content is labeled as \emph{Safe} when that content is suitable for the general public.


% \begin{figure}[!ht]
%   \centering
%   \begin{subfigure}[b]{0.45\textwidth}
%     \centering
%     \includegraphics[width=0.8\textwidth]{img/safe.png}
%     \subcaption{Safe videos.}
%     \label{fig:samples_safe}
%   \end{subfigure}
%   \hspace{2em}
%   \begin{subfigure}[b]{0.45\textwidth}
%     \centering
%     \includegraphics[width=0.8\textwidth]{img/sensitive.png}
%     \subcaption{Sensitive videos }
%     \label{fig:samples_not_save}
%   \end{subfigure}
%   \caption{Examples of Safe and Sensitive content.}
%   \label{fig:samples}
% \end{figure}
\begin{figure*}[!ht]
    \centering
    \includegraphics[width=0.95\textwidth]{img/safe-sensitive-horizontal.png}
    \caption{Examples of safe (top row) and sensitive videos (bottom row).}
    \label{fig:samples}
    \vspace{-1em}
\end{figure*}

Figure \ref{fig:samples} illustrates these two categories.
There are four scenes with safe content on the top row, and four scenes with sensitive content on the bottom row.

%As anyone can easily access any content on the Internet, whether exploring on search engines or through social networks, some groups of people (especially children) are very vulnerable to the exposure of content not suitable for their ages. 
%This situation calls for some media control strategy managed by parents or tutors, ensuring the least exposure to sensitive content.

%Controlling the type of content uploaded to video storage services requires an automatic analysis in an accurate and efficient way. 

%In this work, we created a CNN based model for video feature extraction and validate these video features experimenting with different baseline models to detect sensitive content.
%Then we evaluate the best model in a dataset created with videos sampled from the Brazilian RNP (National Research Network) repository video@RNP.\footnote{\url{https://www.video.rnp.br}}
%In our experimentation, the best model achieves a recall of 94.4\% and an F1-score of 95.6\% for pornography class.

Other works, such as \cite{moreira2019multimodal}, share our motivations and objectives, as described in Section \ref{sec:related}. However, most of them do not use both audio and image for classification. Some use hand-crafted feature extraction methods instead of more recent CNNs that has been showing great potential in video recognition and classification.

Our work uses two CNNs: one to extract image sequence features and the other to extract audio features.
As we get one feature vector for each second of the video, we can approach the feature classification task as a time series classification, using a Recurrent Neural Network (RNN) as baseline. We also can combine those features to create a single feature vector for the entire video, which then is used as the input for other baseline classifiers.
%Our method uses a rather simpler approach for video classification and yet it still yields results significantly close to related works.


%\todo{Also talk about scene localization, scene detection}

%What is the difference between sensitive content detection and classification?

Although the term \textit{classification} can be used to define the task we are addressing, in this work we favor the term \textit{detection} over \textit{classification} to avoid leaving an open interpretation, as the term \textit{classification} is often used to include tasks with multiple classes, while our task relies specifically on binary classification. Furthermore, \textit{Detection} in this context should not be interpreted as the task to find (either time-wise or space-wise) voice content in the video.
We use detection in a more specific sense: the act of finding out if sensitive media is or is not present in the content. 

%investigar a classifciacao e conteudo sensivel em video, principalmente em videos pornograficos e vi9lencia extrema

%objetivos espcificos
%cricao de dataset
%implemwentacao de baselines
%testar a eficiencia da extracao de features ja usada no yt8m
% testar a multimodalidade
% testar abordagem sequencial e não sequencial



%colocar exelpos do que é gore e oq n é 

%Obrigado pela recomendação professora! Foram excelentes leituras e tentarei trazer alguns desses conceitos para o meu trabalho.
% Infelizmente acho que nenhum dos datasets, apesar de numerosos e variados, é compativel com a nossa definição,

% O dataset que mais se aproximou com a nossa definição foi o MediaEval 2015, porém nosso dataset vem de principalmente de gravações amadoras ou CCTV, não de cenas cinematográficas.
% Seria interessante se testar se o treino em cenas amadoras se traduzem em bons resultados em cenas de estúdio, porém para fazer isso teriamos que filtrar alguns vídeos desse dataset que não entram em nossa definição.

% As nossas definições de violência são os vídeos de extremamente violentos (gore), que incluem pelo menos um desses: Sangue, Multilação e Morte.
% Não fazem parte da nossa definição: presença de armas, lutas, discussões, acidentes de carro (que não contenham nenhum dos topicos acima), violência emocional/mental e violência animada/cartunizada.

%In this work our goal is to create and validate a approach for sensitive content detection in video.

% Some of the questions we aim to answer with this work are:
% \begin{enumerate}
%     \item How does this approach compares with the related work?
%     \item What is the impact of also using audio in the model's performance?
%     \item Can the same model have a performance higher than 90\% on both pornography and violence detection tasks?
% \end{enumerate}

%To find the answers to these questions and fulfil our goal, 
The main contributions of this work are:
\begin{enumerate}
    % Criamos um dataset para a tarefa de classificacao de conteudo sensivel, o maior do mundo até onde sabemos
    \item To our knowledge, the largest sensistive content detection dataset.
    \item To our knowledge, we have obtained the best results in this task using only a generalistic feature extraction method and generic classifiers.
    % Testamos baselines nessa tarefa para validar o dataset e o funcionamento da extração de features
    \item We trained and tested baseline classifiers on the features extracted from our dataset in order to validate both the dataset and the efficiency nad operation of the genralistic feature extraction networks.  
    % Experimentamos com diferentes configurações de classificadores, inculsive sequenciais e naõ sequenciais
    \item We compared sequential and non sequential classifiers in this task.
    % Testamos a importancia dos features de imagem e de audio separadamente
    \item We tested the importance of image and audio features in this task.
    \item We also validate our approach by testing our best baseline in a well known pornography detection dataset, 2k-pornography. 
\end{enumerate}
Although the largest dataset for this task by our knowledge, our dataset is not manually labeled, which begs the question if it is noise-less enough for any training and evaluation in this task.  
Our intent is not to replace the 2k-pornography dataset, but to be a complement it, it still is the gold standard for pornography detection, in our dataset the videos were not manually labeled by a human, so we need to validate this dataset.

To perform this task, we created a large scale dataset, extracted features from this dataset using an generalist and well known feature extraction for video classification method, and performed experiments such as compare baseline classification models, compare which type of classification model (sequential or not) performs best, and compared the importance of audio and image features. further detailed in Section \ref{sec:approach}.

This dissertation proposal is organized as follows:
In Section \ref{sec:related} we discuss some of the related work.
Then, in Section \ref{sec:approach}, we detail the proposed method to detect sensitive content in videos.
We present our dataset and metrics in Sections \ref{sec:dataset} and \ref{sec:metrics}, respectively.
Then, we propose experiments to evaluate our models in Section \ref{sec:experiments}.
Finally, in Sections \ref{sec:contrib},~\ref{sec:expdcontrib} and \ref{sec:schedule} we present, respectively, our currently already achieved contributions, our still expected contributions, and our schedule.
%%% -*- coding: utf-8 -*-
\newpage

\chapter{Related Work}
\label{chap:related}

 %O dataset que mais se aproximou com a nossa definição foi o MediaEval 2015, porém nosso dataset vem de principalmente de gravações amadoras ou CCTV, não de cenas cinematográficas.
 
% commented out because wqe have a more recent paper from theese same authors
%Song and Kin~\cite{song2018pornographic} create a scheme for detecting pornography videos using multimodal features: Image descriptor features of the frame sequence, extracted using the VGG-16 CNN\cite{vgg}, motion features extracted using optical flow\cite{opticalflow}, and audio features extracted using a Mel-scaled spectrogram.
%The final features for each model are obtained by an average pooling of each of the features by a sample in the video.
%Each of those kinds of features is used in a single SVM classifier per type of feature, resulting in an image sequence-based detector, a motion-based detector, and an audio-based detector.
%The final decision-making is done by model stacking all detectors.
%The authors used a modified dataset based on the 2k-pornography dataset\cite{2kdataset} for training and testing.
%The results of their method are an average of 63.4\% with a 100\% true positive rate for porn, and an average of 23.5\% of the false-positive rate. 
%Our work also uses multimodal features, but we only use image sequence features and audio features, furthermore, we use an Inception-V3 CNN instead of the VGG-16 CNN for extracting image sequence features and use an Audio VGG CNN for extracting audio features.
%Although the authors achieve 100\% recall rate for pornography, which is the main goal of their task, their model also has a 23.5\% false-positive rate, which means that normal videos would be occasionally classified as pornography.
%Our aim is to also have a true positive rate as high as theirs, while still further reducing the false positive rate.

Castro~\cite{torres2018automatic} shows an implementation of a pornography video classifier using a convolutional neural network from Open pornography~\cite{mahadeokar2016open} and the dataset from Nude Detection in Video using Bag-of-Visual-Features~\cite{lopes2009nude} dataset.
The CNN does a logistic regression on each frame, resulting in a value from 0 to 1 at each frame.
The higher the value is, the higher is the likelihood of the frame being pornography.
The dataset used contained 90 non-pornography video segments and 89 pornography video segments extracted from 11 movies.
The final score for the video the max value from all frames of the video.
The experiment showed an accuracy of 81\%, and a F1-score, and Matthew’s correlation coefficient(MCC) for the pornography class of 0.8047 and 0.6343, respectively.
Although the work also approaches pornography content detection in videos problem with CNN like ours, it does not make use of audio features.
The method is also different, it performs the regression first, then it takes the max value from all frames of the video, while ours, in the non-sequential approach, combines features from all frames of the video into a single vector of features (mainly by averaging) and then performs classification on the resulting features.

% Tem o trabalho do professor novo também, ele usa convnets pra extrair as features e uma rnn pra classificar
%https://www.sciencedirect.com/science/article/pii/S0925231217312493
Wehrmann \textit{et al.}~\cite{wehrmann2018adult} classify adult content trained on the NPDI pornography video dataset ~\cite{avila2013pooling}, which consists of 802 videos, totaling 80 hours of videos, half of them with adult content.
Those videos were processed by keyframes, varying between 1 and 320 frames per video.
The selected keyframes of each video were chosen by a scene segmentation algorithm, resulting in 16727 images. % this was done by the dataset authors, not the work authors
Their architecture consists of a Convolutional Network and an Long-Short Term Memory Network (LSTM)~\cite{hochreiter1997long}.
Those models were chosen for feature extraction with CNN and sequence learning with LSTM, taking into consideration modifications on the images such as scaling and distorting.
Using this approach the authors achieved a score of 95.6\% ± 1 accuracy and 0.990 AU(ROC).
In our model, we also approached the video analysis using frame by frame processing, but but we also processed the extracted sound from each frame.% using a pre-trained Convolutional Neural Network%, instead of an untrained one.
%\hl{Resulting in an accuracy of 98\% and 97.97\% F1-score for pornography class.}

Sing \textit{et. al.}\cite{singh2019kidsguard} proposes a fine-grained approach for child unsafe video representation and detection. One of its main objectives is to optimize detection on sparsely present child unsafe content and it does so by using a VGG16\cite{vgg} Convolutional Neural Network (CNN) to encode each frame, at 1-second granularity, in 512 real values. 
Then an LSTM autoencoder is trained to output the sequence backward on those encoded frames. 
Once the LSTM autoencoder is trained, then a fully connected layer of neurons is used to fine-tune and classify each frame. 
The dataset used comprises 109,835 short-duration video clips extracted from four animes. 
The results for binary classification using safe and unsafe classes were 81\% recall for unsafe and 0.88 AUC(ROC) for unsafe class.
Although this work also has similar objectives as ours and also uses a CNN-based encoding method, ours uses both visual and audio features to encode a video. 
%Their work uses 1 frame per second granularity and ours has the same encoding rate. 
The main difference between both works is on the dataset: Theirs consists of small clips of only anime videos. Ours also uses other types of videos such as live-action and other animations. 
%In our dataset, the length of videos range from 6 seconds to 30 minutes.

Song \textit{et. al.}~\cite{song2020enhanced} proposed a multimodal stacking scheme for fast and accurate online detection of pornographic content.
Their work uses both visual and auditory features as input for their detection method. 
They use a VGG16 model and a bi-directional LSTM to extract visual features and a combination of a Mel-scaled spectrogram followed by multilayered dilated convolutions to extract audio features. 
Using only the visual and auditory features, a video classifier and an audio classifier are trained, respectively. 
By using both features together, one fusion classifier is also trained.
Then, these three component classifiers are combined in an ensemble scheme to reduce the false-negative errors and for faster detection. 
The proposed detection method yields a true positive rate of 95.40\% and a false negative rate of 4.60\% on the pornography class, totaling a recall for the pornography class of 95.40\% and accuracy of 92.33\%. 
The dataset used was the NPDI 2k-pornography\cite{2kdataset} dataset plus examples of videos with only pornographic or non-pornographic audio collected by the authors. 
This work is similar to ours because it also uses a multimodal approach to detection, albeit ours is not for pornography detection only.
It also uses the same sampling rate of a frame for each second and uses a deep learning method for extracting high-level features, which are then classified by one or more machine learning models. 
% Our work has a different dataset, comprised of pornography, gore, and violent videos for the inappropriate class and miscellaneous and educational YouTube videos as an appropriate class. 
We also use different feature extraction methods for image and audio features. 
Finally, in contrast with their ensemble approach, we use a single model to classify the extracted features from our dataset.


Moreira \textit{et.al.}~\cite{moreira2019multimodal} has similar detection focuses as ours: Pornography and Violence. 
Their method uses four multi-modal classifiers, two for audio and two for image, those classifiers were fed features from multiple handcrafted feature extraction methods. Their work is geared towards mobile device applications and also allows for sensitive scene localization.
The authors propose a method for sensitive scene localization which uses the output of four multi-modal classifiers on snippets of the video, then creates a fusion vector at each second of the video. 
Finally, they test different classifiers on the fusion vector for each task: detecting pornography and detecting violence. Their best result on the pornography task was 90.75\% accuracy and 93.53\% on the F2 metric. For the violent videos, they achieved 0.502 on the MAP2014 evaluation metric.
Some differences between this work and our are mainly its objectives: To detect if and at what time the sensitive video occurs. While our only objective is to detect if there is or not sensitive content in a video. Their method is geared towards mobile devices, while ours is geared towards video hosting platforms.
%Our definition of violent videos consists of extremely violent videos, while theirs also considers videos with minor violence such as fights.
Other differences stand out as the dataset and the methods used for feature extraction and classification. 
The Violent Scenes Dataset \cite{VSD2014} is comprised of violent scenes from movies, while ours contains real violent scenes.
We use an authorial dataset and investigate what results a deep learning-based approach to this problem can yield.


%THEIR METHOD EVALUATES ON EACH FRAME, OUR ON THE ENTIRE VIDEO}

Wang \textit{et. al.} ~\cite{wang2019porn} proposes a pornography method for use in live streams, focusing on real-time processing, their work uses multimodal features, namely, image, audio, and optical flow~\cite{horn1981opticalflow}. 
An Xception~\cite{chollet2017xception} model is used to extract spatial features from keyframes. 
To get the optical flow frames, they also use a CNN to extract the optical flow from the video, then, use another Xception model to extract the high-level optical flow features. 
Finally, they use a short-time Fourier transformation to create spectrograms and feed those spectrograms to a third Xception model and thus acquiring the extracted audio features.  
Each of the multimodal features extracted then is passed onto bidirectional GRUs\cite{dey2017GRU}, to obtain temporal context, then, to create a better-unified representation, all the features go through three interconnected Attention-gated layers, each with three Attention-gated units proposed in the paper. After obtaining the dense representation of the input types, it is applied a fully connected layer of neurons with \textit{softmax} function. Their work archives 76.33\% accuracy and runs at 66.1 fps.
In our work, we strive for detecting both violence and pornography, we use only two types of input data, image, and audio, and we use a specific CNN for each type of data, while their work focused only on detecting pornography and used the same CNN model for all three types of input. 
%We investigate whether bigger and more specialized models can create better high-level features and further increase the quality of classification.

Liu \textit{et. al.} ~\cite{liu2020analyzing} proposes a multi-modal approach to pornography detection, it uses audio-frames and visual-frames to create handcrafted low-level features based on, respectively, periodic patterns and salient regions. Once those features are extracted, they use k-means clustering to create audio and visual codebooks. 
Then, low-level audio and visual features of test videos are converted into mid-level semantic histograms via de audio or visual codebook. 
Finally, the histograms are concatenated to represent the video and a periodicity-based video decision algorithm is used to fuse the classification results of multi-modal codebooks and the results of an SVM trained on the concatenated mid-level semantic features train set.
The true positive rate of their approach achieves 96.7\% while the false positive rate is about 10\%.
%Liu \textit{et. al.} detects pornography, they do not detect violence, and also use handcrafted features such as Region Of Interest (ROI) extraction and skin-color segmentation.
%Whereas our approach uses a fully automatic feature extraction method based on CNNs and our feature fusion method consists of just a concatenation followed by a classifier.

Most related works focus on pornography detection alone, while ours aims at detecting either pornographic or violent content. Moreover, some of them only use image-frame features, whereas we use both audio and image-frame features. We also use deep learning feature extraction methods instead of hand-crafted ones. Feature extraction method, classification method and dataset of each related work are available in Table \ref{tab:related}.
Finally, a central difference is our dataset: Ours contains violent scenes and is significantly larger than most datasets used on other related works.

\begin{landscape}
\begin{table}[]
\scriptsize
\centering
\caption{Related work comparative table.}
\begin{tabular}{l|l|l|l|l|l|l|l|l|l}
Paper                                & Task                                                                    & Feature extraction method                                                                                                                                           & Classifier                                                                                 & F2-Score                                                        & Recall  & Accuracy                                                                                 & F1-Score & AUC (ROC) & Dataset                                                                                                                             \\ \hline
Castro \cite{torres2018automatic}           & \begin{tabular}[c]{@{}l@{}}Pornography\\ detection\end{tabular}         & Resnet 50 for image.                                                                                                                                                & Resnet-50                                                                                  & 0,7798                                                          & 0,7640  & 0,8160                                                                                   & 0,8047   & NA        & \begin{tabular}[c]{@{}l@{}}Open pornography\\  + Nude Detection in \\ Video using \\ Bag-of-Visual-Features \\ dataset\end{tabular} \\ \hline
Wehrmann et al. \cite{wehrmann2018adult}    & \begin{tabular}[c]{@{}l@{}}Pornography\\ detection\end{tabular}         & ResNet-101 for image.                                                                                                                                               & LSTM                                                                                       & 0,9520                                                          & 0,9501  & 0,9560                                                                                   & 0,9548   & 0,9900    & \begin{tabular}[c]{@{}l@{}}NPDI pornography\\ video dataset\end{tabular}                                                            \\ \hline
Sing et. al. \cite{singh2019kidsguard}      & \begin{tabular}[c]{@{}l@{}}Sensitive\\ content\\ detection\end{tabular} & VGG16 for image.                                                                                                                                                    & \begin{tabular}[c]{@{}l@{}}LSTM \\ + FC\end{tabular}                                       & NA                                                              & 0,8100  & NA                                                                                       & NA       & 0,8800    & Author (Animes)                                                                                                                     \\ \hline
Song et. al. \cite{song2020enhanced}        & \begin{tabular}[c]{@{}l@{}}Pornography\\ detection\end{tabular}         & \begin{tabular}[c]{@{}l@{}}VGG16\\ + BiLSTM for image;\\ Mel-scaled spectrogram\\ + Multilayered dilated\\ convolutions for audio.\end{tabular}                     & \begin{tabular}[c]{@{}l@{}}Early \\ + Late fusion FC\\ voting\end{tabular}                 & NA                                                              & 95.40\% & 0,9233                                                                                   & NA       & NA        & \begin{tabular}[c]{@{}l@{}}NPDI\\ 2k-pornography\end{tabular}                                                                       \\ \hline
Moreira et.al. \cite{moreira2019multimodal} & \begin{tabular}[c]{@{}l@{}}Sensitive\\ content\\ detection\end{tabular} & \begin{tabular}[c]{@{}l@{}}HOG \\ for image;\\ TRoF \\ for space-temporal description;\\ MFCC and \\ prosodic features\\  for audio.\end{tabular}                   & \begin{tabular}[c]{@{}l@{}}Thresholding \\ (Pornography);\\ SVM\\ (Violence).\end{tabular} & \begin{tabular}[c]{@{}l@{}}93,53\%\\ (Pornography)\end{tabular} & NA      & \begin{tabular}[c]{@{}l@{}}90,75\%\\(Pornography);\\0.502 MAP2014\\(Violence).\end{tabular} & NA       & NA        & \begin{tabular}[c]{@{}l@{}}NPDI\\ 2k-pornography +\\ Violent Scenes Dataset\\ (MediaEval 2014)\end{tabular}                         \\ \hline
Wang et al. \cite{wang2019porn}             & \begin{tabular}[c]{@{}l@{}}Pornography\\ detection\end{tabular}         & \begin{tabular}[c]{@{}l@{}}Xception for image;\\ Optical flow \\ + Xception for motion;\\ Short-time Fourier\\ transformation \\ + Xception for audio.\end{tabular} & \begin{tabular}[c]{@{}l@{}}bidirectional GRUs \\ + Attention \\ + FC\end{tabular}          & NA                                                              & NA      & 76,33\%                                                                                  & NA       & NA        & \begin{tabular}[c]{@{}l@{}}BJUT streamer\\  dataset (Author)\end{tabular}                                                           \\ \hline
Liu et. al. \cite{liu2020analyzing}         & \begin{tabular}[c]{@{}l@{}}Pornography\\ detection\end{tabular}         & \begin{tabular}[c]{@{}l@{}}Skin color detection \\ + Face detection \\ + Salient regions detection \\ + SURF for image.\end{tabular}                                & SVM                                                                                        & NA                                                              & NA      & 96,70\%                                                                                  & NA       & NA        & Author                                                                                                                             
\end{tabular}
\label{tab:related}
\end{table}
\end{landscape}
%%% -*- coding: utf-8 -*-
\newpage

\chapter{Theory and technical background}
\label{chap:theory}
%%% -*- coding: utf-8 -*-
\newpage

\chapter{Sensitive Content Dataset}
\label{chap:dataset}

In this Chapter, we present how our \textit{dataset} was collected, how it is structured, how the features were extracted, and what metrics we recommend for the main task of this dataset.

\section{Dataset Collection and Assembly}\label{sec:dataset-collection}

% falar do balanceamento e do sampling do conjunto de teste
% Falar das talbeas no texto
% colocar o resto das tabelas talvez no appendix
% estatisticas depois de dropar os maiores e os menores videos, sem balanceamento
% usamos o balanceamento dropando apesa porn videos pq os gore sao poucos
% iremos distribuir o dataset na versao balanceada e sem balanceamento
% Comparar nossas escolhas de corte de tempo com o do yt 
% Checked for duplicates by name and size and by fdupes, sem garantias de de não ter subvideos


% Before clipping video based on duration
% Mean:  0:06:18.929511  STD:  0:12:18.237458
% There are 1355 (1.05%) videos longer than 00:30:56
% There are 116 (0.09%) videos shorter than 00:00:05

\subsection{Safe content}\label{subsec:dataset-safe}

For \textit{safe content}, we chose to sample instances from Youtube8M\footnote{\url{https://research.google.com/youtube8m}}~\cite{abu2016youtube}. We chose this dataset because of it's size (8 million videos) and because of the wide variety of video classification challenges it supports. We selected 55.000 random videos inside each of the 24 top-level categories, proportionally to the original dataset distribution. As there was no limit for the sample size, we did not had to use any rules to keep small categories. 

We successfully collected 50,988 Youtube videos with metadata. 4,012 of the 55,000 sampled videos failed to download or were unavailable. 

We also collected 8,663 videos from Youtube, hereby refered as ``cherry-picked" safe videos, those videos were selected for the purpose of incremeting the amount of ``hard" videos, as done in \cite{2kdataset}, which are videos that could possibly be misclassified as sensitive, such as Mixed Martial Arts (MMA), breastfeeding, pool parties, beaches and other videos that have a higher amount of skin exposure. The amount of cherry-picked videos collected are listed by its respective query in Table \ref{tab:non-yt-count}. The collection was made by automated means, a script automatically searched and tried to download all videos from the first 100 result pages of each query.

\subsection{Sensitive content}\label{subsec:dataset-sensitive}

For \textit{sensitive content}, we collected pornography and violent videography (thereafter referred to as \textit{gore}) from websites. 

For the pornography, we to sample videos from the XVideos\footnote{\url{https://info.xvideos.com/db}} database. We chose this source because of the database size (7 million videos) and because of the amount and variety of annotations. In this database, each video has one main tag, totalling 60 main tags, and tags (user-created). 

We sampled 55.000 random videos in each tag in equal proportion to the original distribution in these tags. In particular, to prevent tags with fewer videos from disappearing, we have defined that the minimum sample for each tag is the size of the smallest tag. The smallest tag was 'asmr' with 63 videos.

We successfully collected 54,549 pornography videos with metadata. 451 of the 55,000 sampled videos failed to download or were unavailable. We also collected 10,519 ``hard" videos from XVideos, specifically videos that have low skin exposure, fully dressed people, latex costumes, and cosplay. %8572 of those got successsfully mapped to the xvideos database and have metadata

For the gore content, we used a web crawler to extract 2,356 gore videos from various websites dedicated to gore media, such as, BestGore\footnote{\url{https://www.bestgore.com/}} and GoreBrasil\footnote{\url{https://www.gorebrasil.com}}. As these videos were the harder to find and collect, we collected all available videos from each website, no sampling method was applied. Most videos did not stay online more than one week. As there was no contact with the videos and the videos did not have any tags, all metadata collected were the title of the videos. 

Not all video's features were successfully extracted for multiple reasons, such as, corrupt data, unkown format, and missing audio. For those videos with missing audio or image, the features were still generated, but their respective modal feature were zeros. Those videos which do not had any features successfully extracted were removed from the dataset.

We also removed any duplicated videos that were detected, for duplicate video detection we used, we matched either id, title or checksum. 

\section{Dataset Structure}\label{sec:dataset-structure}

Our \textit{dataset} is structured into two main classes (or macro-classes): ``safe'' videos and sensitive videos. The sensitive macro-class is composed of two micro-classes: Pornography and Gore. The safe class is composed of videos videos from Youtube. Finally, each of the micro-classes has main-tags, which are the same main tags from their original metadata in the website, if available. Each instance (a video) of the dataset may also have tags, which are a list of tags that represent the video. The general structure and organization of our dataset is represented in Figure \ref{fig:dataset-structure}.

\begin{figure}[!ht]
    \centering
    \includegraphics[width=0.95\textwidth]{img/dataset/Dataset-Structure-color.png}
    \caption{Dataset tree structure}
    \label{fig:dataset-structure}
\end{figure}


% Videos:
% Total dataset: 127075
% Total size of videos (calculated individually): 3.5TiB
% Total duration of videos (calculated indiviadually): 11806:21:13
% Improper videos: 67424
% Total size of videos (calculated individually): 1.2TiB
% Total duration of videos (calculated indiviadually): 6953:27:41
% Proper videos: 59651
% Total size of videos (calculated individually): 2.2TiB
% Total duration of videos (calculated indiviadually): 4852:53:31
% Total size of features: '896.3GiB'

There are 59,651 safe videos and 67,424 videos with sensitive content. Table \ref{tab:general-stats} presents the general statistics of our dataset, such as total duration (hours, minutes and seconds) of all videos, total (uncompressed) features size, tag coverage (If a video has at least one tag, videos may also have tags but no main tag). If played in real time, a person would take approximately 1 year and 127 days straight to flag all videos in this dataset.

\begin{table}
\centering
\caption{General statistics of the two main classes of the dataset.}
\begin{tabular}{c|r|r} 
\multicolumn{1}{l|}{} & Sensitive  & Safe        \\ 
\hline
Video Count           & 67424      & 59651       \\ 
\hline
Total Duration        & 6953:27:41 & 4852:53:31  \\ 
\hline
Mean Duration         & 00:06:11   & 00:04:52    \\ 
\hline
STD Duration          & 00:04:12   & 00:03:26    \\ 
\hline
Max Duration          & 00:30:55   & 00:30:55    \\ 
\hline
Min Duration          & 00:00:05   & 00:00:05    \\ 
\hline
Total Size            & 1,2TiB     & 2,2TiB      \\ 
\hline
Mean Size             & 19,3MiB    & 39,0MiB     \\ 
\hline
STD Size              & 35,4MiB    & 42,3MiB     \\ 
\hline
Features Size         & 519,4GiB   & 376,8GiB    \\ 
\hline
Tag coverage          & 63036      & 59651       \\ 
\hline
Tag coverage (\%)     & 93,4919    & 100,0000     \\
\end{tabular}
\label{tab:general-stats}
\end{table}

%Youtube
The Youtube micro-class has 25 main tags, as presented in Table \ref{tab:yt-granular}, in Appendix \ref{chap:appendix}, each of the videos in the "hard safe videos" main tag, has one tag, which was the query word used to collect it.

%Porn
The Pornography micro-class has 60 main tags, as presented in \ref{tab:porn-granular}, Appendix \ref{chap:appendix}, these main tags were defined by the database creators, the tags were user created. The instances in this micro-class also have user-created titles.
% Gore
There are no main-tags or tags (but the 'gore' tag) on the Gore micro-class as they were not available on their site. All instances, however, have user-created titles.

\begin{table}
\centering
\caption{Granular statistics of the dataset: Videos collected from Youtube, pornographic videos, and gore videos.}
\begin{tabular}{c|r|r|r} 
\multicolumn{1}{l|}{} & \multicolumn{1}{c|}{Pornography} & \multicolumn{1}{c|}{Gore} & \multicolumn{1}{c}{YouTube}  \\ 
\hline
Video Count           & 65068                     & 2356                      & 59651                         \\ 
\hline
Total Duration        & 6900:17:38                & 53:10:02                  & 4852:53:31                    \\ 
\hline
Mean Duration         & 00:06:21                  & 00:01:21                  & 00:04:52                      \\ 
\hline
STD Duration          & 00:04:10                  & 00:01:26                  & 00:03:26                      \\ 
\hline
Max Duration          & 00:30:55                  & 00:16:56                  & 00:30:55                      \\ 
\hline
Min Duration          & 00:00:05                  & 00:00:05                  & 00:00:05                      \\ 
\hline
Total Size            & 1,2TiB                    & 15,8GiB                   & 2,2TiB                        \\ 
\hline
Mean Size             & 19,8MiB                   & 6,9MiB                    & 39,0MiB                       \\ 
\hline
STD Size              & 35,9MiB                   & 13,9MiB                   & 42,3MiB                       \\ 
\hline
Features Size         & 515,3GiB                  & 4,1GiB                    & 376,8GiB                      \\ 
\hline
Tag coverage          & 63036                     & 0                      & 59651                         \\ 
\hline
Tag coverage (\%)     & 96,8771                   & 0                  & 100,0000                       \\
\end{tabular}
\label{tab:granular-stats}
\end{table}

By using macro and micro classes of this dataset, our dataset  also supports other tasks, other than binary classification of sensitive content, such as:
\begin{itemize}
    \item Multi label classification (or tagging) of pornographic videos;
    \item Multi label classification of "Safe" (Videos that do not contain sensitive content);
    \item Binary classification of extremely violent (gore) videos;
    \item Binary classification of pornography.
\end{itemize}


\subsection{Dataset Distribution}\label{sec:distribution}

The dataset will be distributed as extracted and processed visual and audio features from the videos. Each instance (features from a video) is associated with a id, a label, and a sequence size. We will not distribute raw video data, but we are open and plan to include other feature extraction methods in our dataset. General details on the dataset distribution are available in Appendix \ref{chap:datasheet}.



\subsection{Dataset Balancing}\label{sec:balancing}

For experimenting, we equally balanced both main labels (sensitive/improper and safe/proper), so that both main classes have the same number of instances. One could also choose not to balance both classes equally, since our main metric already take label imbalance into account.
Additionally, when removing excess sensitive content (while balancing), we removed only pornography videos in order to not lower the amount of gore videos.


\subsection{Dataset splits and Test sets}\label{sec:testsets}

We hold out our dataset for testing  our approach: 10\% of the safe videos, then 10\% of gore videos, and sample a number of pornography to to match the amount of safe videos minus the amount of gore test samples, so that the test subset has a balanced amount of sensitive and safe videos, while keeping a valid amount of gore videos. For the micro-classes that have multiple main tags (Youtube and Pornography), we took stratified samples based on the number of each main tag in the dataset. The number of instance by micro-class sampled is presented in Table \ref{tab:subset-stats}.

\begin{table}
\centering
\caption{Test subset statistics.}
\begin{tabular}{l|l|l|l}
                  & Pornography & Gore     & YouTube   \\ \hline
Video Count       & 5732        & 236      & 5968      \\ \hline
Total Duration    & 574:03:29   & 05:22:20 & 443:35:36 \\ \hline
Mean Duration     & 00:06:00    & 00:01:21 & 00:04:27  \\ \hline
STD Duration      & 00:03:44    & 00:01:48 & 00:02:32  \\ \hline
Max Duration      & 00:30:29    & 00:16:56 & 00:29:01  \\ \hline
Min Duration      & 00:00:05    & 00:00:07 & 00:00:07  \\ \hline
Total Size        & 80,2GiB     & 1,5GiB   & 204,8GiB  \\ \hline
Mean Size         & 14,3MiB     & 6,6MiB   & 35,1MiB   \\ \hline
STD Size          & 25,6MiB     & 14,4MiB  & 35,0MiB   \\ \hline
Features Size     & 42,6GiB     & 424,1MiB & 34,5GiB   \\ \hline
Tag coverage      & 5695        & 0        & 5968      \\ \hline
Tag coverage (\%) & 99,3545     & 0,0000   & 100,0000 
\end{tabular}
\label{tab:subset-stats}
\end{table}

As a complementary test dataset, we selected the NPDI 2k-pornography dataset~\cite{avila2013pooling}, it contains 1000 non-pornographic videos and 1000 pornographic videos. Those non-pornographic videos are comprised of ``hard'' and ``easy'' videos according to the likelihood of misclassification. Some examples of ``hard'' videos are those with high amounts of exposed skin, such as swimming and sumo fighting videos. Its general statistics are shown in Table \ref{tab:2kdataset-stats}.

\begin{table}
\centering
\caption{NPDI 2k-pornography dataset statistics.}
\begin{tabular}{c|r|r} 
\multicolumn{1}{l|}{} & \multicolumn{1}{c|}{Porn} & \multicolumn{1}{c}{Non-Porn}  \\ 
\hline
Video Count           & 1000                      & 1000                           \\ 
\hline
Total Duration        & 100:30:32                 & 40:26:06                       \\ 
\hline
Mean Duration         & 00:06:01                  & 00:02:25                       \\ 
\hline
STD Duration          & 00:05:49                  & 00:02:17                       \\ 
\hline
Max Duration          & 00:33:40                  & 00:20:16                       \\ 
\hline
Min Duration          & 00:00:05                  & 00:00:02                       \\ 
\hline
Total Size            & 26,4GiB                   & 18,5GiB                        \\ 
\hline
Mean Size             & 27,0MiB                   & 18,9MiB                        \\ 
\hline
STD Size              & 31,1MiB                   & 21,9MiB                        \\ 
\hline
Features Size         & 7,6GiB                    & 3,1GiB                         \\ 
\hline
Tag coverage          & 0                      & 0                           \\ 
\hline
Tag coverage (\%)     & 0                       & 0                            \\
\end{tabular}
\label{tab:2kdataset-stats}
\end{table}

\section{Metrics}\label{sec:metrics}

% in binary classification, recall of the positive class is also known as “sensitivity”; recall of the negative class is “specificity”.
%https://en.wikipedia.org/wiki/Sensitivity_and_specificity
%https://en.wikipedia.org/wiki/Precision_and_recall

To evaluate each experiment and our approach, we will use Precision (P), Recall (R) and, most importantly, the weighted F2-score. In this section we present a contextualized explanation of these metrics.

% FROM: de morereira et al 2019
%For assessing the performance of the pornography locators, we re-port thenormalized classification accuracyrate (ACC), and theF2measure(F2). Prior to explaining ACC, we need to definerecallandspecificityfrom the point of view of pornography localization.  Specificity, in turn, measures the capacity of alocator to correctly identify truly negative video seconds as so. A spe-cificity of only 50%, for example, means the system mislabels one inevery two seconds of non-pornographic content, wrongly identifying itas sensitive. In this vein, ACC is the mean of recall and specificity. Ahigher accuracy indicates a higher capability of separating porno-graphic video seconds from non-pornographic ones.F2measure, in turn, is a more complex metric that depends also onthe concept ofprecision. From the point of view of pornography loca-lization, precision expresses how many seconds are truly relevant (i.e.,pornographic), among all the ones that a locator identifies as such.Therefore, F2is the weighted harmonic mean of recall and precision,which gives twice more weight to recall than to precision, by means of a=β2parameter.Eq. (7)depicts the original Fβformula=+ ×××+Fβprecision  recallβ   precision  recall(1    ),β22(7)in which we use=β2. In doing so, F2lets us pay more attention to therecall of the solutions, rather than to their precision. This is usefulbecause, in the case of pornographyfiltering, false-negative answers areworse than the false-positive ones. It is less prejudicial to wrongly denythe access to non-pornographic content, than to wrongly disclose por-nographic content. Hence, we can consider that a solution with higherF2measure is better, because it cares more about how many porno-graphic video seconds are really beingfiltered out (recall), instead ofhow many“supposedly”positive seconds are indeed pornographic(precision)



In the context of sensitive content detection, \textit{true positives} are videos predicted as sensitive and are in fact, sensitive. Likewise, \textit{true negatives} are videos predicted as safe and are indeed safe. \textit{False positives} are videos predicted as sensitive, but were safe, the same goes for \textit{false negatives}, which are videos that were predicted as safe, but were actually sensitive.

% from the daniel moreira paper:
% thenormalized classification accuracyrate (ACC)ACC is the mean of recall and specificity. Ahigher accuracy indicates a higher capability of separating porno-graphic video seconds from non-pornographic ones

Precision (Equation~\ref{equation:precision}) measures how many videos predicted as sensitive (both true positives and false positives) are truly sensitive. The Recall (Equation~\ref{equation:recall})  measures how many truly positive videos were correctly identified.

\begin{multicols}{2}
  \begin{equation}
    \label{equation:precision}
    P = \frac{TP}{TP + FP}
  \end{equation}

  \begin{equation}
    \label{equation:recall}
    R = \frac{TP}{TP + FN}
  \end{equation}
  
\end{multicols}

Where $TP, TN, FP$, and $FN$ denote the examples that are true positives, true negatives, false-positives, and false negatives, respectively.

\begin{equation}
\label{equation:fbeta}
F_\beta = (1+\beta^2) \times \frac{P \times R}{(\beta^2 \times P) + R}
\end{equation}

The $F_\beta$-score, defined in Equation~\ref{equation:fbeta}, evaluates the classifier by the harmonic mean between Precision and Recall. To account for label imbalance, after calculating the F2-score metrics for each label, we find their average weighted by support (the number of true instances for each label). 

Most related works, such as~\cite{moreira2019multimodal,wehrmann2018adult,torres2018automatic}, use either F1-score ($\beta=1$) or F2-score ($\beta=2$) metrics as their main evaluation metric. While the F1-score represents an balanced performance metric, the F2-score gives twice more weight to the recall than to precision, which means that the metric is more focused on the recall of a solution.

% in the case of pornographyfiltering, false-negative answers areworse than the false-positive ones. It is less prejudicial to wrongly denythe access to non-pornographic content, than to wrongly disclose por-nographic content. Hence, we can consider that a solution with higherF2measure is better, because it cares more about how many porno-graphic video seconds are really beingfiltered out (recall), instead ofhow many“supposedly”positive seconds are indeed pornographic(precision)
In this work, the F2-score represents an overall performance metric, while the precision and recall metrics can give insights on what the classifier model is doing better and what to improve. We chose the weighted F2-score as our main evaluation metric because when detecting sensitive content it is more important to predict a truly sensitive video than to predict a safe video as sensitive.

%%% -*- coding: utf-8 -*-
\newpage

\chapter{Method}
\label{chap:method}
In this section we detail our method for sensitive content detection in video. We split our approach into three parts: feature extraction, feature fusion and feature classification, as illustrated in Figure \ref{fig:model}.
\begin{figure*}[!ht]
    \centering
    \includegraphics[width=0.9\textwidth]{img/model-2.png}
    \caption{Bimodal architecture for NSFW video classification.}
    \label{fig:model}
    \vspace{-1em}
\end{figure*}

In the feature extraction stage, firstly we split the frames and audio from the video; then, for each media, we use a CNN to extract the features (or embeddings) from each simultaneous video segment. In the second stage, Feature Fusion, we concatenate both audio and frame features. If the classification model is not sequential, we also aggregate the features in this stage. Finally, in the feature classification stage we feed one of the classification models to be experimented with.

\section{Video Embeddings Extraction}
\label{sec:video_features}

CNNs tend to learn low-level features (\textit{e.g.}, in the visual domain: edges, corners, contours) at their first layers. At the intermediate and final layers, the combination of these features helps to extract more complex features, resulting in a vector of continuous values, referred to as \textit{embeddings}, that might be used for classification and other tasks. In this work, we use two benchmark CNNs to extract both image and audio \textit{embeddings} by using a transfer learning technique~\cite{tan2018survey}.


%Afterwards, we apply PCA (+ whitening) to reduce feature dimensions to 1024, followed by quantization (1 byte per coefficient).
%These two compression techniques reduce the size of the data by a factor of 8. The mean vector and covariance matrix for PCA was computed on all frames from the Train partition. We quantize each 32-bit float into 256 distinct values (8 bits) using optimally computed (non-uniform) quantization bin boundaries. We confirmed that the size reduction does not significantly hurt the evaluation metrics. In fact, training all baselines on the full-size data (8 times larger than what we publish), increases all evaluation metrics by less than 1%.
\pva{Falar do pca e da quantizacao?}

By using the feature extraction method created for the Youtube-8m benchmark, we can test an feature extraction method that is powerful enough to represent features that can be in multiple tasks, such as multi-label video classification, video recommendation, and human activity recognition.

"Since the video-level representations are unsupervised (extracted independently of the labels), these representations are far less specialized to the labels associated with the current dataset, and can generalize better to new tasks or video domains."~\cite{abu2016youtube}

In order to validate our dataset, we used the same feature extraction method used in the Youtube-8m dataset challenges~\cite{abu2016youtube}, both networks were pre-trained and frozen. They were not retrained for application sensitive content classification. Which gives future works an opportunity to develop even more efficient and smaller feature extraction networks for this specific task.


As described in ~\cite{abu2016youtube},
To generate image frame features and audio features we decode each video at approximately 1 frame-per-second. For the image frame features, we used an InceptionV3  network~\cite{szegedy2016rethinking} pre-trained on the ImageNet\footnote{\url{http://www.image-net.org/}} dataset. We also use of a AudioVGG~\cite{hershey2017cnn} network with pre-trained weights in the Audioset\footnote{\url{https://research.google.com/audioset/}} dataset to extract the audio embeddings.

Each of these CNNs were used as published by their authors; the only modification was the removal of classification layers in both CNNs to obtain their respective embeddings.

\section{Feature Fusion}
\label{sec:feature_fusion}

Once we have the features from both image and audio, we should make a decision about which method is best to fuse the information from these different domains. Snoek et al. \cite{snoek2005featurefusion} presents two main strategies for information fusion in semantic video analysis: \emph{Early fusion} methods, which works directly with the extracted features, and \emph{Late fusion} methods, which operates on classification outputs from specialized models.
%\pva{For a more recent survey about data fusion and multimedia retrieval, please refer to \cite{jiang2013features}.}
In this work, because we have high abstraction level features, we opted to investigate the most simple approach, which is to use a single model on the concatenated features from both media inputs.

Because of this, in order to create the final embeddings, we concatenate both image and audio embeddings extracted in the same frame and audio window. This generates a sequence of the same size of the number of seconds of the video. After this concatenation, each time-step has 1,152 features: 128 audio features and 1024 frame features.

Notice that with this approach, the video is transformed into a time series, and to use it in non-sequential models (\textit{e.g.}~SVM, KNN, and MLP) we need to turn this sequence into a single feature vector that represents the whole video. In our setting, we did that by taking the average, median, standard deviation, min, and max values for each feature to represent the entire video. In summary, we turn the sequence of features with size $n$ and shape $n$ by 1,152 into a single feature with shape 1 by 5,760.

\section{Classifiers}
\label{sec:classifiers}

% \pva[inline]{porque foi escolhidos esses modelos de classficação, qual a relação deles com o nosso tipo de dado de vídeo.}

For the feature classification task, we will investigate both sequential models (which use the extracted embeddings in a time series format), and non-sequential ones (which use a single aggregated embeddings vector).
We want to experiment with both approaches in order to investigate if a more compact format, such as the single embeddings vector, can yield results at least as good (or even better) than the full feature sequence data.
%Furthermore, we can test if a sequential model can outperform a non-sequential model in specific cases that demand long term memory, such as long videos with very small sensitive scenes.
As an example, one can think of a long video that has a pornographic scene in one second out of its entirety. 
In a non-sequential representation of the extracted features, this short pornographic fragment could be left ``hidden'' among the other non-pornographic frames of the video, as illustrated in Figure \ref{fig:model-non-sequence}.
\begin{figure*}[!ht]
    \centering
    \includegraphics[width=0.9\textwidth]{img/model-non-sequence.png}
    \caption{Sequential features with aggregation, the sensitive scene (red) might vanish among the the other scenes during aggregation.}
    \label{fig:model-non-sequence}
    \vspace{-1em}
\end{figure*}
In a sequential representation, although time series classifiers usually output a prediction after reading the entire sequence, the embedding vectors of each second of the video would not be aggregated and thus could be analysed section by section, as illustrated in Figure \ref{fig:model-sequence}.

\begin{figure*}[!ht]
    \centering
    \includegraphics[width=0.9\textwidth]{img/model-sequence.png}
    \caption{Sequential features with no aggregation. In a output after reading the entire sequence, this can also be susceptible to information vanishing.}
    \label{fig:model-sequence}
    \vspace{-1em}
\end{figure*}
Although a sequential representation contains possibly much more redundant data than the non-sequential one, it could give the sequential classification model an important edge of detail over the less granular non-sequential ones.
% To do the feature classification task, we experimented four well known classification mohave dels, one using extracted video features in the time series format, and three that use a single aggregated feature vector.

For the sequential classification model, we chose the Long Short-Term Memory (LSTM)\cite{hochreiter1997long} networks.
It has been a commonly used time series classification baseline model. %\pva{mayber say we'll test GRUs?}

For the non-sequence models, we chose Support Vector Machines (SVM)~\cite{cortes1995support}, K-Nearest Neighbors (KNN)~\cite{peterson2009k}, and Multilayer Perceptron (MLP)~\cite{haykin2009neural}.
Among all of the experimented models, the \textit{Support Vector Machine (SVM)} is the most used in the literature.
% \begin{enumerate}[leftmargin=*]
% \item 
It is a classification model in which the data is mapped into a higher dimension input space, where an optimal separating hyper-plane is constructed.
% These decision surfaces are found by solving a linearly constrained quadratic programming problem.
% \item 
The second model, \textit{K-Nearest Neighbors (KNN)} uses distance measure between training samples so that the k-nearest neighbors always belong to the same class, while samples from different classes are separated by a large margin. 
It was chosen because it used also by related work, although it is a simple classification method.
% \item 
The third model is the \textit{Multilayer-perceptron (MLP)}, which contains layers of nodes: an input layer, an output layer and various hidden layers in between. 
This one was selected because it is also commonly used as a final classifier on deep neural networks.  
% The number of layers used is problem dependent, as is the number of nodes in each hidden layer.
% The weights are adjusted by local optimization using a set of feature vectors so that the network produces the optimal expected output.
% \item 
%Lastly, \textit{Long short-term memory (LSTM)}, different than the feed-forward neural networks, process the entire sequences of data using feedback connections.
% \end{enumerate}
For model evaluation, we performed 20-fold cross validation for all baseline models.
%
\section{Proposed Analysis}\label{sec:experiments}

We evaluate the performances of baseline classifiers over the video \textit{embeddings} that were extracted from our dataset, described in Chapter \ref{chap:dataset}. Then, we choose the best performing classifier during validation stage and test its performance on the \textit{test sets}.  
%\pva{maybe add a image for the process?}
We designed a set of cases that might help us find insights and assess the performance and shortcomings of our dataset and approach.  

% In this work our goal is to create and validate a approach for sensitive content detection in video.

% Some of the questions we aim to answer with this work are:
% \begin{enumerate}
%     \item How does this approach compares with the related work?
%     \item What is the impact of also using audio in the model's performance?
%     \item Can the same model have a performance higher than 90\% on both pornography and violence detection tasks?
% \end{enumerate}

Our objective with these analysis is to attest the quality of our dataset and approach at detecting sensitive content on video.

We did not perform extensive hiperparameter optimization on the baseline models, we performed most hiperparameters changes on the SVM model, since it is the model most sensitive to hiperparameters optimization.
%, while also answering our research questions, stated in Section \ref{sec:introduction}.
% In this analysis, our objective is to attest to the quality of our video \textit{embeddings}.
 
% In the next subsections, we discuss the analysis setup, used metrics, and our findings. In Subsection \ref{sec:config} we describe the training configuration for each model.
% Next, in Subsection \ref{sec:metrics} we describe the evaluation metrics.
% And finally, in Subsection \ref{sec:results} we present our empirical findings.
\begin{enumerate}[start=0,label={(\bfseries E\arabic*):}]
%\item The sensitive content detection task: This is the main analysis of this work, in this analysis we test the capabilities of our approach and the best performing classification model on our test subset.%\pva{maybe referenciar ao subset de teste do nosso dataset?}

%\item The pornography detection task: In this analysis we evaluate our approach and the best performing classification model on the pornography detection task using our test subset. 

%\item The gore detection task: In this analysis we evaluate our approach and the best performing classification model on the gore detection task using our test subset.

\item Testing only on image features: In this analysis we evaluate our approach on the our test subset using the visual (frames) features only.

\item Testing only on audio features: In this analysis we evaluate our approach on the our test subset using the audio features only.

%\item Testing on the pornography-2k: In this analysis we evaluate our approach on the pornography-2k dataset.

\item Testing pornography using audio only videos: In this analysis we evaluate our approach on the pornography-2k dataset using the audio features only.

\item Testing pornography using only on image features: In this analysis we evaluate our approach on the pornography-2k dataset using the visual features only.
%\item esting MEDIAEVAL Violent Scenes Dataset

\item Investigate misclassified videos in the test sets: In this analysis case we search for insights on what videos our approach fails to correctly detect sensitive content.
% \item sequence classification models vs non-sequence models
\end{enumerate}

In the next Chapter, we present and discuss the results of our baselines and report each analysis.
%%% -*- coding: utf-8 -*-
\newpage

\chapter{Results}
\label{chap:results}

Having performed 20-fold cross validation, we collected all metrics through all the folds. Table \ref{tab:cross-validation-results} presents mean, standard deviation, min and max. The full results for each fold are available on Table \ref{tab:full-folds}, Appendix \ref{chap:appendix}.

\begin{table}[!ht]
\centering
\caption{Weighted F2-Score (in percentage) for each model across 20-Fold Cross Validation).}
\begin{tabular}{l|l|l|l|l}
      & \multicolumn{1}{c|}{MLP} & \multicolumn{1}{c|}{LSTM} & \multicolumn{1}{c|}{SVM} & \multicolumn{1}{c}{KNN} \\ \hline
count & 20,0000                  & 20,0000                   & 20,0000                  & 20,0000                 \\ \hline
mean  & \textbf{99,0743}                  & 98,9899                   & 98,8993                  & 96,4738                 \\ \hline
std   & 0,1349                   & 0,1174                    & 0,1347                   & 0,2915                  \\ \hline
min   & 98,7945                  & 98,7943                   & 98,5445                  & 95,6932                 \\ \hline
25\%  & 99,0073                  & 98,9175                   & 98,8499                  & 96,3585                 \\ \hline
50\%  & 99,0962                  & 98,9848                   & 98,9174                  & 96,4108                 \\ \hline
75\%  & 99,1708                  & 99,0684                   & 98,9756                  & 96,6564                 \\ \hline
max   & 99,2885                  & 99,1915                   & 99,0836                  & 96,9448                
\end{tabular}
\label{tab:cross-validation-results}
\end{table}

Comparing the models in Table \ref{tab:cross-validation-results} the model with the highest Weighed F2-Score is the Multilayer Perceptron model. Although it has the second highest standard deviation, it is still small enough such that the smallest variation still is higher than the second best model, the Long-Short-Term-Memory model.\pva{adicionar teste de diferença significativa}

\begin{figure*}[!ht]
    \centering
    \includegraphics[width=0.5\textwidth]{img/results/boxplot.png}
    \caption{Boxplot of the results of each model throughout the 20-fold cross validation.}
    \label{fig:results-boxplot}
\end{figure*}

\begin{figure*}[!ht]
    \centering
    \includegraphics[width=0.9\textwidth]{img/results/histogram.png}
    \caption{Histogram of the results of each model throughout the 20-fold cross validation.}
    \label{fig:results-histogram}
\end{figure*}

Figures \ref{fig:results-boxplot} and \ref{fig:results-histogram}, show the difference in the distribution of results of each model. The simplest model, K-Nearest Neighbors, has the most difference from the models. The three other models, however, are relatively close.

We tested the best performing model, the Multilayer-Perceptron, on the test subset, shown in Table \ref{tab:test-general-results}. 

\begin{table}[!ht]
\centering
\caption{Test subset results, shown in absolute values).}
\begin{tabular}{l|l|l|l|l|l}
             & precision & recall & f1-score & f2-score & support                         \\ \hline
Safe         & 0,9895    & 0,9906 & 0,9900   & 0,9897   & 5973,0000                       \\ \hline
Sensitive    & 0,9906    & 0,9895 & 0,9900   & 0,9904   & 5973,0000                       \\ \hline
weighted avg & 0,9900    & 0,9900 & 0,9900   & 0,9900   & \multicolumn{1}{l}{11946,0000}
\end{tabular}
\label{tab:test-general-results}
\end{table}

As shown in Table \ref{tab:test-general-results} and in Figure \ref{fig:cf-test}, the MLP model has performed within the range of the mean of the cross validation. It can also be noted that, the most frequent errors were \textit{false positives}, when the model predicted a video as Sensitive when it was in fact, Safe. 

\begin{figure*}[!ht]
    \centering
    \includegraphics[width=0.49\textwidth]{img/results/MLP-TEST.png}
    \caption{Confusion matrix of the predictions of the best model in the test subset.}
    \label{fig:cf-test}
\end{figure*}

We also tested our best model in each sub task, pornography and gore binary classification. For the pornography, shown in Table \ref{tab:test-porn-results}, and Figure \ref{fig:cf-test-porn}, the most frequent error were the \textit{false negatives}, in which the model predicted that the most samples were predicted as Safe, but were actually Sensitive. 

\begin{table}[!ht]
\centering
\caption{Results testing pornography only, shown in absolute values).}
\begin{tabular}{l|l|l|l|l|l}
             & precision & recall & f1-score & f2-score & support    \\ \hline
Safe         & 0,9947    & 0,9902 & 0,9925   & 0,9939   & 5737,0000  \\ \hline
Sensitive    & 0,9903    & 0,9948 & 0,9925   & 0,9911   & 5737,0000  \\ \hline
weighted avg & 0,9925    & 0,9925 & 0,9925   & 0,9925   & 11474,0000
\end{tabular}
\label{tab:test-porn-results}
\end{table}

For the gore, shown in Table \ref{tab:test-gore-results}, and Figure \ref{fig:cf-test-gore}, the most frequent error were the \textit{false positive}, in which the model predicted that the most samples were predicted as Sensitive, but were actually Safe.  

\begin{table}[!ht]
\centering
\caption{Results testing gore videos only, shown in absolute values).}
\begin{tabular}{l|l|l|l|l|l}
             & precision & recall & f1-score & f2-score & support  \\ \hline
Safe         & 0,8764    & 0,9915 & 0,9304   & 0,8834   & 236,0000 \\ \hline
Sensitive    & 0,9902    & 0,8602 & 0,9206   & 0,9661   & 236,0000 \\ \hline
weighted avg & 0,9333    & 0,9258 & 0,9255   & 0,9248   & 472,0000
\end{tabular}
\label{tab:test-gore-results}
\end{table}

\begin{figure*}[!ht]
    \centering
    \begin{subfigure}[b]{0.49\textwidth}
        \includegraphics[width=0.93\textwidth]{img/results/MLP-TEST-PORN.png}
        \caption{Confusion matrix of the model on the pornography videos of the test subset.}
        \label{fig:cf-test-porn}
    \end{subfigure}
    \begin{subfigure}[b]{0.49\textwidth}
        \includegraphics[width=0.90\textwidth]{img/results/MLP-TEST-GORE.png}
        \caption{Confusion matrix of the model on the gore videos of the test subset.}
        \label{fig:cf-test-gore}
    \end{subfigure}
    \caption{Confusion matrices of the best performing model on the pornography and gore subsets.}
\end{figure*}

To evaluate if our model and our dataset on the pornography detection (binary classification) task, we also tested our best performing baseline model on a well known dataset for pornography detection: The 2k-pornography dataset. The results are shown in \ref{tab:test-2k-results} and in Figure \ref{fig:cf-test-2k}. The most common errors were false negatives, in which the model predicts the instance as a Safe, but the true label were Sensitive.

\begin{table}[!ht]
\centering
\caption{Test on the NPDI 2k-pornography dataset results, shown in absolute values).}
\begin{tabular}{l|l|l|l|l|l}
             & precision & recall & f1-score & f2-score & support    \\ \hline
Safe         & 0,9665    & 0,8080 & 0,8802   & 0,9411   & 1.000,0000 \\ \hline
Sensitive    & 0,8351    & 0,9720 & 0,8983   & 0,8354   & 1.000,0000 \\ \hline
weighted avg & 0,9008    & 0,8900 & 0,8893   & 0,8883   & 2.000,0000
\end{tabular}
\label{tab:test-2k-results}
\end{table}

\begin{figure*}[!ht]
    \centering
    \includegraphics[width=0.49\textwidth]{img/results/MLP-2K-TEST.png}
    \caption{Confusion matrix of the predictions of the best model in the NPDI 2k-pornography dataset.}
    \label{fig:cf-test-2k}
\end{figure*}


\section{Discussion}\label{sec:discussion}
% Falar dos resultados de teste e de cada task
% Para tarefas de porn, o mais comum foram falsos negativos, como esperado
% Para tarefas  de gore, o mais comum foram falsos positivos, que previram como sensivel mas eram safe, que é o idela, mas pq esses videos sao bem parecidos com safe
% Falar dos erros do pornography2k, que usou f2 score, de quanto foi o modelo deles, e dos erros mais comuns (Falsos negativos) 
% Falar pq isso aconteceu, qual a diferença entre os dois datasets e os modelos
% por eles usarem late fusion talvez tenha ajudado, especialmente pq o nosso modelo usa só as features de imagem
\section{Analysis cases}\label{sec:experiments-discussion}

In order to further investigate the impact of each multi-modal feature in our best performing model, the Multilayer Perceptron, we tested it on our test subset and on the NPDI 2k-pornography dataset, but only using one modal feature at a time. For example, in Figure \ref{fig:cf-test-image}, we tested the MLP model using only visual (frames) features, specifically, were changed all audio features for zero to simulate a video with no audio features.

\begin{figure*}[!ht]
    \centering
    \begin{subfigure}[b]{0.49\textwidth}
        \includegraphics[width=0.94\textwidth]{img/results/MLP-TEST-IMAGE-ONLY.png}
        \caption{Confusion matrix of the model on the test subset using only image features.}
        \label{fig:cf-test-image}
    \end{subfigure}
    \begin{subfigure}[b]{0.49\textwidth}
        \includegraphics[width=0.94\textwidth]{img/results/MLP-TEST-AUDIO-ONLY.png}
        \caption{Confusion matrix of the model on the test subset using only audio features.}
        \label{fig:cf-test-audio}
    \end{subfigure}
    \caption{Confusion matrices of the model on the test subset using only one multi-modal feature at a time.}
\end{figure*}

As observed in Figures \ref{fig:cf-test-image} and \ref{fig:cf-test-audio}, our model had the same performance with only visual features, but mis-classified all Sensitive videos. This means that the MLP model ignored all audio features for all videos. It is relying only in visual features, even though there are examples of videos in the dataset, in which the main feature of a sensitive video is audio. % ASMr VIDEOS

\begin{figure*}[!ht]
    \centering
    \begin{subfigure}[b]{0.49\textwidth}
        \includegraphics[width=0.94\textwidth]{img/results/MLP-2K-TEST-IMAGE-ONLY.png}
        \caption{Confusion matrix of the model on the NPDI 2k-pornography dataset using only image features.}
        \label{fig:cf-test-2k-image}
    \end{subfigure}
    \begin{subfigure}[b]{0.49\textwidth}
        \includegraphics[width=0.94\textwidth]{img/results/MLP-2K-TEST-AUDIO-ONLY.png}
        \caption{Confusion matrix of the model on the NPDI 2k-pornography dataset using only audio features.}
        \label{fig:cf-test-2k-audio}
    \end{subfigure}
    \caption{Confusion matrices of the model on the NPDI 2k-pornography dataset using only one multi-modal feature at a time.}
\end{figure*}

We confirmed this pattern with the test with the NDPI 2k-pornography dataset, as shown in Figures \ref{fig:cf-test-2k-image} and \ref{fig:cf-test-2k-audio}.%, our model relied only in visual features.

% Discutir vantagens e desvantagens do eraly fusion por causa de forçar o modelo a aprender uma mdeia de cada e depois combinar
% Possibilidades pra so ter aprendido imagens:
%Diferença no tamanho das deatures de audio e frames
%Diferença pra early e late fusion
%Só a mlp aprendeu isso
%Features ruins de audio
% Receptive field pra áudio pequeno
% Audio n faz diferença ***: perguntar pra paulo oq ele quis dizer

%%% -*- coding: utf-8 -*-
\newpage

\chapter{Conclusions}
\label{chap:conclusions}

In this work we created a large scale video dataset for sensitive content detection and a multi-modal approach to sensitive content detection in video. 
%Our model uses both audio and visual features. 
It uses pre-trained convolutional neural networks, and applies a late-fusion feature method, which is simpler than the early fusion approach since we use a single model to classify both features.

We evaluated our models by testing on a test subset and on a popular dataset. We expected to validate our dataset and approach while maintaining a similar performance to the existing methods.

It is important to note that our approach is not focused on mobile platforms, therefore memory and disk space were not major constraints. 

It is worth mentioning that our results on the sensitive content detection are not directly comparable to the related works since our definitions of violence do not match them. However, we could compare our approach on the pornography detection sub-task by testing our best performing baseline model on the NPDI 2k-pornography dataset. Our approach yielded and F2-Score of 86.83\%, compared to our related work, \pva{colocar valor do trabalho relacionado e citar}

\section{Currently published papers}
\label{sec:contrib}

By the the time of this writing, three papers about this work have already been published in conferences: \cite{2019NSFWbaseline}, \cite{2020PornDetectionSBIE}, and \cite{shouldisee}. We also have finished the construction of the dataset for sensitive content detection, to be published soon.

\section{Future Work}
\label{sec:future}


% após o fim desse trabalho é claro que ainda muitos locais em nossa abordagem para melhoria, mas somente com uma extraçaõ de features generalista e baselines, já conseguimos alcançar abordagens estado da arte
%imagino que a partir daqui, o sentido do progresso do trabalho é diminuir e simpleificar a extração de features, ao mesmo tempo especializando, ou seja, treinando a extração de features para a prórpia tarefa
Even with generic feature extraction CNNs we archieved almost 90\% on the pornography detection task. 
One future work is to create a late fusion model and evaluate it based on each feature type.
Another possible future work is to add motion information, such as optical flow, to the dataset.
Another possibility is to extend this approach even further, creating new CNNs from scratch to classify the videos based on, audio, visual and motion features.
Test if a sequential model can outperform a non-sequential model in specific cases that demand long term memory, such as long videos with very small sensitive scenes.

\section{Acknowledgements}
\label{sec:acks}
This work was supported by the Artificial Intelligence challenge, created by Brazil's National Research Net (RNP) and Microsoft, in 2019.

\arial
\bibliography{references}

% Apendix chapters below.
\normalfont
\appendix
%%% -*- coding: utf-8 -*-
\newpage

\chapter{Complementary tables}\label{chap:appendix}

\begin{table}[]
\centering
\caption{Weighed F2-Score for each fold and each baseline model.}
\begin{tabular}{c|l|l|l|l}
\multicolumn{1}{l|}{} & MLP     & LSTM    & SVM     & KNN     \\ \hline
\textbf{1}            & 99,1196 & 98,9038 & 98,9917 & 96,2230 \\ \hline
\textbf{2}            & 99,1017 & 99,0298 & 99,0513 & 96,6345 \\ \hline
\textbf{3}            & 99,1915 & 99,1374 & 98,6160 & 96,4204 \\ \hline
\textbf{4}            & 99,0477 & 98,9758 & 98,7941 & 96,3366 \\ \hline
\textbf{5}            & 99,1735 & 98,9938 & 99,0710 & 96,7749 \\ \hline
\textbf{6}            & 99,0118 & 98,8861 & 98,9128 & 96,2796 \\ \hline
\textbf{7}            & 99,0119 & 99,1196 & 98,8926 & 96,5178 \\ \hline
\textbf{8}            & 99,1374 & 99,1915 & 98,8679 & 96,4022 \\ \hline
\textbf{9}            & 98,8501 & 98,7963 & 98,8498 & 96,4195 \\ \hline
\textbf{10}           & 99,0398 & 98,9439 & 98,9219 & 96,3810 \\ \hline
\textbf{11}           & 98,9937 & 98,9220 & 98,8284 & 96,9448 \\ \hline
\textbf{12}           & 99,2273 & 98,9938 & 98,5445 & 96,3657 \\ \hline
\textbf{13}           & 99,2810 & 99,1375 & 99,0836 & 96,9391 \\ \hline
\textbf{14}           & 99,1103 & 98,9526 & 98,9756 & 96,2957 \\ \hline
\textbf{15}           & 99,0908 & 99,0316 & 98,9038 & 96,4567 \\ \hline
\textbf{16}           & 99,1700 & 99,1502 & 98,9399 & 96,3819 \\ \hline
\textbf{17}           & 98,7945 & 98,9525 & 98,9398 & 96,3807 \\ \hline
\textbf{18}           & 98,8929 & 98,8338 & 98,9756 & 96,9062 \\ \hline
\textbf{19}           & 99,2885 & 99,0513 & 98,9756 & 96,7222 \\ \hline
\textbf{20}           & 98,9517 & 98,7943 & 98,8499 & 95,6932
\end{tabular}
\label{tab:full-folds}
\end{table}

\begin{table}
\centering
\caption{The amount of youtube videos collected per query.}
\begin{tabular}{l|r} 
Query  & \multicolumn{1}{l}{Video count}  \\ 
\hline
amamentacao    & 987                               \\ 
\hline
animation      & 823                               \\ 
\hline
breastfeeding  & 724                               \\ 
\hline
ufc            & 592                               \\ 
\hline
model          & 541                               \\ 
\hline
pool           & 526                               \\ 
\hline
gymnastics     & 474                               \\ 
\hline
pool party     & 459                               \\ 
\hline
ecchi          & 431                               \\ 
\hline
fisiculturismo & 426                               \\ 
\hline
boxing         & 416                               \\ 
\hline
yoga           & 368                               \\ 
\hline
animação       & 348                               \\ 
\hline
anime          & 337                               \\ 
\hline
surf           & 321                               \\ 
\hline
MMA            & 314                               \\ 
\hline
swimming       & 297                               \\ 
\hline
beach          & 279                               \\ 
\hline
Total          & 8663                              \\
\end{tabular}
\label{tab:non-yt-count}
\end{table}

\begin{table}[]
\centering
%\hspace{-10em}
\caption{Video distribution per main tag on the Youtube macro-class.}
\begin{tabular}{l|l|l|l|l|l|l}
\begin{tabular}[c]{@{}l@{}}Main\\ tag\end{tabular} & \begin{tabular}[c]{@{}l@{}}\#\\ Videos\end{tabular} & \begin{tabular}[c]{@{}l@{}}Total\\ Duration\end{tabular} & \begin{tabular}[c]{@{}l@{}}Mean\\ Duration\end{tabular} & \begin{tabular}[c]{@{}l@{}}STD\\ Duration\end{tabular} & \begin{tabular}[c]{@{}l@{}}Total\\ Size\end{tabular} & \begin{tabular}[c]{@{}l@{}}Tag\\ coverage\\ (\%)\end{tabular} \\ \hline
Hard Safe Videos                                   & 8640                                                & 1278:28:34                                               & 00:08:52                                                & 00:06:46                                               & 638.5GiB                                             & 100                                                           \\ \hline
Arts \& Entertainment                              & 8554                                                & 598:33:35                                                & 00:04:11                                                & 00:01:36                                               & 221.5GiB                                             & 100                                                           \\ \hline
Games                                              & 7182                                                & 530:11:13                                                & 00:04:25                                                & 00:01:47                                               & 250.6GiB                                             & 100                                                           \\ \hline
Autos \& Vehicles                                  & 6812                                                & 455:38:54                                                & 00:04:00                                                & 00:01:39                                               & 253.3GiB                                             & 100                                                           \\ \hline
(Unknown)                                          & 4054                                                & 283:05:42                                                & 00:04:11                                                & 00:01:37                                               & 109.0GiB                                             & 100                                                           \\ \hline
Food \& Drink                                      & 3552                                                & 262:25:03                                                & 00:04:25                                                & 00:01:42                                               & 129.2GiB                                             & 100                                                           \\ \hline
Sports                                             & 3053                                                & 203:38:24                                                & 00:04:00                                                & 00:01:38                                               & 100.4GiB                                             & 100                                                           \\ \hline
Business \& Industrial                             & 2584                                                & 14:38:23                                                 & 00:04:14                                                & 00:01:43                                               & 83.1GiB                                              & 100                                                           \\ \hline
Computers \& Electronics                           & 2325                                                & 169:45:31                                                & 00:04:22                                                & 00:01:44                                               & 84.3GiB                                              & 100                                                           \\ \hline
Hobbies \& Leisure                                 & 2110                                                & 149:44:31                                                & 00:04:15                                                & 00:01:44                                               & 77.1GiB                                              & 100                                                           \\ \hline
Pets \& Animals                                    & 2000                                                & 129:44:49                                                & 00:03:53                                                & 00:01:37                                               & 61.9GiB                                              & 100                                                           \\ \hline
Shopping                                           & 1667                                                & 22:32:11                                                 & 00:04:15                                                & 00:01:41                                               & 57.8GiB                                              & 100                                                           \\ \hline
Home \& Garden                                     & 1543                                                & 104:37:46                                                & 00:04:04                                                & 00:01:40                                               & 46.9GiB                                              & 100                                                           \\ \hline
Science                                            & 1233                                                & 82:13:19                                                 & 00:04:00                                                & 00:01:32                                               & 32.5GiB                                              & 100                                                           \\ \hline
Beauty \& Fitness                                  & 848                                                 & 59:25:57                                                 & 00:04:12                                                & 00:01:40                                               & 29.3GiB                                              & 100                                                           \\ \hline
Travel                                             & 688                                                 & 22:03:23                                                 & 00:04:00                                                & 00:01:37                                               & 21.1GiB                                              & 100                                                           \\ \hline
Law \& Government                                  & 658                                                 & 45:25:34                                                 & 00:04:08                                                & 00:01:42                                               & 19.3GiB                                              & 100                                                           \\ \hline
Internet \& Telecom                                & 427                                                 & 05:49:49                                                 & 00:04:11                                                & 00:01:37                                               & 14.5GiB                                              & 100                                                           \\ \hline
Books \& Literature                                & 362                                                 & 01:19:18                                                 & 00:04:11                                                & 00:01:36                                               & 7.9GiB                                               & 100                                                           \\ \hline
People \& Society                                  & 295                                                 & 20:40:09                                                 & 00:04:12                                                & 00:01:43                                               & 7.4GiB                                               & 100                                                           \\ \hline
Reference                                          & 290                                                 & 21:15:52                                                 & 00:04:23                                                & 00:01:44                                               & 8.8GiB                                               & 100                                                           \\ \hline
News                                               & 253                                                 & 16:40:53                                                 & 00:03:57                                                & 00:01:36                                               & 6.4GiB                                               & 100                                                           \\ \hline
Jobs \& Education                                  & 235                                                 & 17:01:50                                                 & 00:04:20                                                & 00:01:39                                               & 6.7GiB                                               & 100                                                           \\ \hline
Finance                                            & 196                                                 & 15:30:18                                                 & 00:04:44                                                & 00:01:49                                               & 4.8GiB                                               & 100                                                           \\ \hline
Real Estate                                        & 70                                                  & 04:52:48                                                 & 00:04:10                                                & 00:01:34                                               & 1.9GiB                                               & 100                                                           \\ \hline
Health                                             & 20                                                  & 01:29:33                                                 & 00:04:28                                                & 00:01:39                                               & 563.7MiB                                             & 100                                                          
\end{tabular}
\label{tab:yt-granular}
\end{table}


%
\tiny
\begin{table}[]
%\resizebox{10cm}{!}{
\centering
%\hspace{-10em}
\caption{Video distribution per main tag on the Pornography macro-class.}
\scriptsize
\begin{tabular}{l|l|l|l|l|l|l}
\begin{tabular}[c]{@{}l@{}}Main\\ tag\end{tabular} & \begin{tabular}[c]{@{}l@{}}\#\\ Videos\end{tabular} & \begin{tabular}[c]{@{}l@{}}Total\\ Duration\end{tabular} & \begin{tabular}[c]{@{}l@{}}Mean\\ Duration\end{tabular} & \begin{tabular}[c]{@{}l@{}}STD\\ Duration\end{tabular} & \begin{tabular}[c]{@{}l@{}}Total\\ Size\end{tabular} & \begin{tabular}[c]{@{}l@{}}Tag\\ coverage\\ (\%)\end{tabular} \\ \hline
gay                                                & 20604                                               & 1888:35:01                                               & 00:05:29                                                & 00:03:26                                               & 170.7GiB                                             & 100,0000                                                      \\ \hline
teen                                               & 18303                                               & 1922:49:21                                               & 00:06:18                                                & 00:03:20                                               & 312.9GiB                                             & 100,0000                                                      \\ \hline
blowjob                                            & 4197                                                & 503:44:56                                                & 00:07:12                                                & 00:03:00                                               & 88.5GiB                                              & 100,0000                                                      \\ \hline
other                                              & 3348                                                & 348:48:55                                                & 00:06:23                                                & 00:06:16                                               & 47.0GiB                                              & 91,6502                                                       \\ \hline
cumshot                                            & 2333                                                & 348:27:39                                                & 00:08:57                                                & 00:06:34                                               & 66.6GiB                                              & 100,0000                                                      \\ \hline
hard\_porn                                         & 1947                                                & 256:37:28                                                & 00:07:54                                                & 00:04:26                                               & 116.0GiB                                             & 0,0000                                                        \\ \hline
anal                                               & 1821                                                & 248:31:25                                                & 00:08:11                                                & 00:06:19                                               & 83.0GiB                                              & 100,0000                                                      \\ \hline
lesbian                                            & 1269                                                & 150:05:51                                                & 00:07:05                                                & 00:04:02                                               & 26.4GiB                                              & 100,0000                                                      \\ \hline
sexy                                               & 1058                                                & 112:49:57                                                & 00:06:23                                                & 00:05:03                                               & 19.9GiB                                              & 100,0000                                                      \\ \hline
amateur                                            & 1021                                                & 85:53:07                                                 & 00:05:02                                                & 00:04:46                                               & 25.5GiB                                              & 100,0000                                                      \\ \hline
milf                                               & 804                                                 & 91:00:05                                                 & 00:06:47                                                & 00:04:47                                               & 15.0GiB                                              & 100,0000                                                      \\ \hline
bdsm                                               & 800                                                 & 92:24:09                                                 & 00:06:55                                                & 00:04:02                                               & 33.8GiB                                              & 100,0000                                                      \\ \hline
shemale                                            & 753                                                 & 67:58:58                                                 & 00:05:25                                                & 00:04:14                                               & 8.4GiB                                               & 100,0000                                                      \\ \hline
exotic                                             & 672                                                 & 69:31:01                                                 & 00:06:12                                                & 00:05:15                                               & 11.5GiB                                              & 100,0000                                                      \\ \hline
big\_tits                                          & 586                                                 & 68:57:58                                                 & 00:07:03                                                & 00:04:03                                               & 19.1GiB                                              & 100,0000                                                      \\ \hline
ass                                                & 571                                                 & 53:05:27                                                 & 00:05:34                                                & 00:04:51                                               & 23.0GiB                                              & 100,0000                                                      \\ \hline
sex\_toys                                          & 547                                                 & 71:14:52                                                 & 00:07:48                                                & 00:04:34                                               & 24.6GiB                                              & 100,0000                                                      \\ \hline
asian\_woman                                       & 469                                                 & 50:16:47                                                 & 00:06:25                                                & 00:05:20                                               & 16.2GiB                                              & 100,0000                                                      \\ \hline
lingerie                                           & 416                                                 & 55:24:42                                                 & 00:07:59                                                & 00:03:42                                               & 17.6GiB                                              & 100,0000                                                      \\ \hline
cam\_porn                                          & 411                                                 & 52:02:43                                                 & 00:07:35                                                & 00:05:27                                               & 9.9GiB                                               & 100,0000                                                      \\ \hline
stockings                                          & 397                                                 & 52:09:06                                                 & 00:07:52                                                & 00:03:49                                               & 18.1GiB                                              & 100,0000                                                      \\ \hline
blonde                                             & 362                                                 & 51:42:48                                                 & 00:08:34                                                & 00:05:49                                               & 16.3GiB                                              & 100,0000                                                      \\ \hline
bukkake                                            & 279                                                 & 30:40:28                                                 & 00:06:35                                                & 00:06:14                                               & 6.4GiB                                               & 100,0000                                                      \\ \hline
interracial                                        & 274                                                 & 30:32:53                                                 & 00:06:41                                                & 00:04:13                                               & 3.3GiB                                               & 100,0000                                                      \\ \hline
big\_ass                                           & 237                                                 & 21:03:45                                                 & 00:05:19                                                & 00:05:04                                               & 6.7GiB                                               & 100,0000                                                      \\ \hline
orgy                                               & 196                                                 & 21:24:24                                                 & 00:06:33                                                & 00:02:38                                               & 3.3GiB                                               & 100,0000                                                      \\ \hline
latina                                             & 170                                                 & 14:26:20                                                 & 00:05:05                                                & 00:04:54                                               & 3.4GiB                                               & 100,0000                                                      \\ \hline
pornstar                                           & 162                                                 & 18:52:39                                                 & 00:06:59                                                & 00:04:53                                               & 4.0GiB                                               & 100,0000                                                      \\ \hline
toons                                              & 162                                                 & 16:46:25                                                 & 00:06:12                                                & 00:05:43                                               & 4.4GiB                                               & 100,0000                                                      \\ \hline
brunette                                           & 154                                                 & 21:10:59                                                 & 00:08:15                                                & 00:05:36                                               & 4.2GiB                                               & 100,0000                                                      \\ \hline
solo\_-\_masturbation                              & 137                                                 & 12:26:16                                                 & 00:05:26                                                & 00:05:09                                               & 2.1GiB                                               & 100,0000                                                      \\ \hline
pissing                                            & 135                                                 & 15:43:48                                                 & 00:06:59                                                & 00:04:42                                               & 4.1GiB                                               & 100,0000                                                      \\ \hline
massage                                            & 133                                                 & 12:26:51                                                 & 00:05:36                                                & 00:01:49                                               & 1.0GiB                                               & 100,0000                                                      \\ \hline
squirting                                          & 126                                                 & 13:07:19                                                 & 00:06:14                                                & 00:04:40                                               & 2.1GiB                                               & 100,0000                                                      \\ \hline
creampie                                           & 125                                                 & 14:52:00                                                 & 00:07:08                                                & 00:06:17                                               & 3.4GiB                                               & 100,0000                                                      \\ \hline
heels                                              & 114                                                 & 14:07:21                                                 & 00:07:25                                                & 00:05:17                                               & 2.4GiB                                               & 100,0000                                                      \\ \hline
virtual\_reality                                   & 113                                                 & 10:51:17                                                 & 00:05:45                                                & 00:03:09                                               & 1.5GiB                                               & 100,0000                                                      \\ \hline
feet                                               & 110                                                 & 10:11:12                                                 & 00:05:33                                                & 00:04:07                                               & 1.6GiB                                               & 100,0000                                                      \\ \hline
fisting                                            & 109                                                 & 12:20:43                                                 & 00:06:47                                                & 00:05:16                                               & 1.4GiB                                               & 100,0000                                                      \\ \hline
indian                                             & 108                                                 & 07:11:21                                                 & 00:03:59                                                & 00:04:29                                               & 564.3MiB                                             & 100,0000                                                      \\ \hline
facial                                             & 108                                                 & 15:44:01                                                 & 00:08:44                                                & 00:06:05                                               & 2.5GiB                                               & 100,0000                                                      \\ \hline
mature                                             & 105                                                 & 11:14:26                                                 & 00:06:25                                                & 00:05:20                                               & 1.4GiB                                               & 100,0000                                                      \\ \hline
gapes                                              & 104                                                 & 09:30:18                                                 & 00:05:29                                                & 00:03:27                                               & 787.8MiB                                             & 100,0000                                                      \\ \hline
oiled                                              & 103                                                 & 09:28:37                                                 & 00:05:31                                                & 00:04:16                                               & 1.0GiB                                               & 100,0000                                                      \\ \hline
big\_cock                                          & 102                                                 & 10:51:22                                                 & 00:06:23                                                & 00:04:23                                               & 1.4GiB                                               & 100,0000                                                      \\ \hline
sex\_dolls                                         & 101                                                 & 04:40:37                                                 & 00:02:46                                                & 00:03:21                                               & 969.2MiB                                             & 100,0000                                                      \\ \hline
black\_woman                                       & 100                                                 & 06:29:52                                                 & 00:03:53                                                & 00:04:48                                               & 1.0GiB                                               & 100,0000                                                      \\ \hline
redhead                                            & 100                                                 & 11:25:17                                                 & 00:06:51                                                & 00:05:37                                               & 2.9GiB                                               & 100,0000                                                      \\ \hline
bi\_sexual                                         & 97                                                  & 11:59:56                                                 & 00:07:25                                                & 00:04:47                                               & 1.5GiB                                               & 100,0000                                                      \\ \hline
bbw                                                & 96                                                  & 05:14:33                                                 & 00:03:16                                                & 00:04:02                                               & 803.4MiB                                             & 100,0000                                                      \\ \hline
workout                                            & 95                                                  & 08:36:48                                                 & 00:05:26                                                & 00:02:58                                               & 981.9MiB                                             & 100,0000                                                      \\ \hline
shaved\_pussy                                      & 95                                                  & 09:49:53                                                 & 00:06:12                                                & 00:04:44                                               & 1.1GiB                                               & 100,0000                                                      \\ \hline
gangbang                                           & 93                                                  & 12:12:51                                                 & 00:07:52                                                & 00:06:46                                               & 1.3GiB                                               & 100,0000                                                      \\ \hline
celebrity                                          & 93                                                  & 05:02:23                                                 & 00:03:15                                                & 00:04:00                                               & 661.2MiB                                             & 100,0000                                                      \\ \hline
real\_amateur                                      & 91                                                  & 09:26:47                                                 & 00:06:13                                                & 00:06:53                                               & 1.3GiB                                               & 100,0000                                                      \\ \hline
japanese                                           & 89                                                  & 13:46:47                                                 & 00:09:17                                                & 00:06:21                                               & 2.0GiB                                               & 100,0000                                                      \\ \hline
swingers                                           & 82                                                  & 05:35:26                                                 & 00:04:05                                                & 00:05:38                                               & 529.5MiB                                             & 100,0000                                                      \\ \hline
ass\_to\_mouths                                    & 76                                                  & 11:56:49                                                 & 00:09:25                                                & 00:08:03                                               & 2.8GiB                                               & 100,0000                                                      \\ \hline
arab                                               & 72                                                  & 04:37:33                                                 & 00:03:51                                                & 00:03:28                                               & 1.2GiB                                               & 100,0000                                                      \\ \hline
asmr                                               & 63                                                  & 08:09:12                                                 & 00:07:45                                                & 00:05:57                                               & 4.2GiB                                               & 100,0000                                                     
\end{tabular}
\label{tab:porn-granular}
\end{table}

\bibliographystyle{bibstyles/IEEEtran}

\end{document}
% \documentclass[11pt]{article}

% \usepackage[pdftex]{graphicx} % OU
% % \usepackage[dvips]{graphicx}

% % \usepackage[round]{natbib}
% \usepackage{adjustbox,array,url,hyperref,multirow,makecell,tabularx,caption,subcaption}
% % \usepackage{url,hyperref,multirow,makecell}
% \usepackage[table]{xcolor}
% \usepackage{lipsum}
% \usepackage{cite}
% \usepackage{amsfonts}
% %\usepackage[latin1]{inputenc}
% \usepackage[utf8]{inputenc}
% %\usepackage[brazilian,brazil]{babel}
% %\usepackage{apacite}
% %\usepackage{float}
% \usepackage{booktabs}
% \usepackage{multirow}
% \usepackage{multicol}
% \usepackage{enumerate}
% \usepackage[shortlabels]{enumitem}

% \usepackage{todonotes}
% \newcommand{\pva}[1]{\color{blue}\textbf{TODO: }#1\color{blue}}

% \usepackage{tikz}
% \usetikzlibrary{
%   angles,
%   arrows,
%   arrows.meta,
%   calc,
%   intersections,
%   positioning,
%   quotes,
%   shapes.geometric,
%   through,
% }
% \tikzset{
%   x=0em,
%   y=0em,
%   node distance=1.2em,
%   >=stealth',
% }

% \newcommand*\rot[1]{\rotatebox{90}{#1}}

% \newtheorem{proposition}{Proposition}
% \newtheorem{lemma}{Lemma}
% \newtheorem{theorem}{Theorem}

% \newcommand{\CQD}{\mbox{\rule[0pt]{1.5ex}{1.5ex}}\medskip}
% \newcommand{\M}{\mathcal{M}}

% \setlength{\oddsidemargin}{0.5cm} \setlength{\textwidth}{15cm}
% \setlength{\topmargin}{-1.5cm} \setlength{\textheight}{22.3cm}%{24.7cm}

% \usepackage{pdfpages}

% \begin{document}

% \LARGE

% % \bigskip

% %% TITLE
% \begin{center}
% {\bf Sensitive Content Detection in Video with Deep Learning
% \Large
% \\Dissertation}
% \end{center}


% \bigskip
% \normalsize

% \begin{flushright}

% Pedro Vincius Almeida de Freitas\\
% Pontifical Catholic University of Rio de Janeiro\\
% \texttt{pedropva@telemidia.puc-rio.br}
% \end{flushright}


% \date{}

% % $~$ \\

% \thispagestyle{empty}

% %
% %
% Workaround for keywords. The keywords in the catalographic sheet must be separated by dots, while the ones shown in the abstract must be separated by semi-colons.
% Thats why we have two commands for each language: \keywords declares the keywords for the catalographic sheet, while \keywordsabstract declares the ones for the abstract.
\keywords
{
  \key{Conteúdo Sensível;}
  \key{Detecção de Conteúdo Sensível;}
  \key{Classificação Multimodal de Videos;}
  \key{Deep Learning.}
}

\keywordsabstract
{
  \key{Conteúdo Sensível;}
  \key{Detecção de Conteúdo Sensível;}
  \key{Classificação Multimodal de Videos;}
  \key{Deep Learning.}
}

\keywordsuk
{
  \key{Sensitive Content;}
  \key{Sensitive Video Dataset;}
  \key{Multimodal Video Classification;}
  \key{Deep Learning.}
}

\keywordsabstractuk
{
  \key{Sensitive Content;}
  \key{Sensitive Video Dataset;}
  \key{Multimodal Video Classification;}
  \key{Deep Learning.}
}

\abstract{
Grandes quantidades de vídeo são carregadas em plataformas de hospedagem de vídeo a cada minuto. Esse volume de dados apresenta um desafio no controle do tipo de conteúdo enviado para esses serviços de hospedagem de vídeo, pois essas plataformas são responsáveis por qualquer mídia sensível enviada por seus usuários.
Nesta dissertação, definimos conteúdo sensível como sexo, violencia fisica extrema, gore ou cenas potencialmente pertubadoras ao espectador. Apresentamos um conjunto de dados de vídeo sensível para classificação binária de vídeo (se há conteúdo sensível no vídeo ou não), contendo 127 mil vídeos anotados, cada um com seus embeddings visuais e de áudio extraídos. Também treinamos e avaliamos quatro modelos baseline para a tarefa de detecção de conteúdo sensível em vídeo. O modelo com melhor desempenho obteve 99\% de F2-Score ponderado no nosso subconjunto de testes e 88,83\% no conjunto de dados NPDI pornography-2k.
}

\abstractuk{
Massive amounts of video are uploaded on video-hosting platforms every minute. This volume of data presents a challenge in controlling the type of content uploaded to these video hosting services, for those platforms are responsible for any sensitive media uploaded by their users.
There has been an abundance of research on methods for developing automatic detection of sensitive content. In this dissertation, we define sensitive content as sex, extreme physical violence, gore or any scenes potentially disturbing to the viewer. We present a sensitive video dataset for binary video classification (whether there is sensitive content in the video or not), containing 127 thousand tagged videos, Each with their extracted audio and visual embeddings. We also trained and evaluated four baseline models on the sensitive content detection in video task. The best performing model archieved 99\% weighed F2-Score on our test subset and 88.83\% on the NPDI pornography-2k dataset.
}


% % \newpage
% \pagenumbering{roman} \setcounter{page}{-1}

% % TABLE OF CONTENTS -  OPTIONAL
% %\newpage
% \tableofcontents

% %% BEGIN DOCUMENT
% \newpage
% \pagenumbering{arabic} \setcounter{page}{1}


% %%% Sections....
% 
% Workaround for keywords. The keywords in the catalographic sheet must be separated by dots, while the ones shown in the abstract must be separated by semi-colons.
% Thats why we have two commands for each language: \keywords declares the keywords for the catalographic sheet, while \keywordsabstract declares the ones for the abstract.
\keywords
{
  \key{Conteúdo Sensível;}
  \key{Detecção de Conteúdo Sensível;}
  \key{Classificação Multimodal de Videos;}
  \key{Deep Learning.}
}

\keywordsabstract
{
  \key{Conteúdo Sensível;}
  \key{Detecção de Conteúdo Sensível;}
  \key{Classificação Multimodal de Videos;}
  \key{Deep Learning.}
}

\keywordsuk
{
  \key{Sensitive Content;}
  \key{Sensitive Video Dataset;}
  \key{Multimodal Video Classification;}
  \key{Deep Learning.}
}

\keywordsabstractuk
{
  \key{Sensitive Content;}
  \key{Sensitive Video Dataset;}
  \key{Multimodal Video Classification;}
  \key{Deep Learning.}
}

\abstract{
Grandes quantidades de vídeo são carregadas em plataformas de hospedagem de vídeo a cada minuto. Esse volume de dados apresenta um desafio no controle do tipo de conteúdo enviado para esses serviços de hospedagem de vídeo, pois essas plataformas são responsáveis por qualquer mídia sensível enviada por seus usuários.
Nesta dissertação, definimos conteúdo sensível como sexo, violencia fisica extrema, gore ou cenas potencialmente pertubadoras ao espectador. Apresentamos um conjunto de dados de vídeo sensível para classificação binária de vídeo (se há conteúdo sensível no vídeo ou não), contendo 127 mil vídeos anotados, cada um com seus embeddings visuais e de áudio extraídos. Também treinamos e avaliamos quatro modelos baseline para a tarefa de detecção de conteúdo sensível em vídeo. O modelo com melhor desempenho obteve 99\% de F2-Score ponderado no nosso subconjunto de testes e 88,83\% no conjunto de dados NPDI pornography-2k.
}

\abstractuk{
Massive amounts of video are uploaded on video-hosting platforms every minute. This volume of data presents a challenge in controlling the type of content uploaded to these video hosting services, for those platforms are responsible for any sensitive media uploaded by their users.
There has been an abundance of research on methods for developing automatic detection of sensitive content. In this dissertation, we define sensitive content as sex, extreme physical violence, gore or any scenes potentially disturbing to the viewer. We present a sensitive video dataset for binary video classification (whether there is sensitive content in the video or not), containing 127 thousand tagged videos, Each with their extracted audio and visual embeddings. We also trained and evaluated four baseline models on the sensitive content detection in video task. The best performing model archieved 99\% weighed F2-Score on our test subset and 88.83\% on the NPDI pornography-2k dataset.
}

% % DEFESA PEDRO - Principais questões levantadas pela banca

% 1) Prof Sandra

% - Não achou clara a motivação do pipeline
% - Sugeiriu categorização dos trabalhos relacionados e tabela comparativa
% - Motivar melhor o PCA e extrair a variância
% - Melhorar a explicação da 1a abordagem
% - Usar como base o artigo "Datasheets for Datasets" para documentar como foi montado o dataset e como ele deve ser usado
% - Fazer análises sobre a importância do áudio e do vídeo e ter conclusões sobre o uso em conjunto


% 2) Alberto

% - Colocar os agradecimentos microsoft e RNP explicando
% - não são 6 experimentos e sim análises


% 3) Julio

% - Extração de features para áudio: por que usar CNNs? 
% - Justificar melhor o PCA
% - Tabela comparativa dos trabalhos relacionados
% - un unicco trabalho de redes recorrents. Testar GRU (?)
% - COmo vai fazer a análise dos erros? 
% - Explicabilidade (?)

\section{Introduction}
\label{sec:introduction}

% INTRO SLR
The amount of multimedia content on the internet is increasing each year.
More than 300 hours of video are uploaded to YouTube every minute.\footnote{\url{https://biographon.com/youtube-stats}}
In this context, studies have shown that about 56\% of children between 10 and 13 years old have a smartphone \cite{remosoftware,chollet2017xception}, and 8 out of 10 teenagers have had a friend who shared some sensitive media through social networks such as Facebook, Twitter, and Whatsapp.\footnote{https://www.netnanny.com/the-importance-of-parental-control/}


% In Brazil, the ``Cicarely case'' was an example that forced youtube to be blocked.\footnote{\url{http://g1.globo.com/Noticias/Tecnologia/0,,AA1412609-6174-363,00.html}}
% In our research, we are interested in helping to avoid scenarios where pornography can be uploaded to education channels, which might expose students, sometimes underage, to this kind of content.\footnote{\url{https://g1.globo.com/sp/sao-paulo/noticia/2020/06/19/professor-de-etec-na-zona-norte-de-sp-e-afastado-apos-se-masturbar-durante-aula-virtual.ghtml}}. 
%This scenery presents challenges on controlling which kind of contents are uploaded to this storage and distribution services, while dealing with great amounts of videos.

This huge amount of data sharing pattern presents a challenge to the control of the type of content that is loaded to these video repositories. By allowing the upload of sensitive content from malicious users, content providers become exposed to legal issues. This is also a problem for users in those platforms, as they might get exposed to this content without a warning.

Methods based on \textit{Deep Learning} (DL) became the \textit{state-of-the-art} in various segments related to automatic video analysis. More specifically, 
Convolutional Neural Networks (CNN) architectures, or ConvNets, have become the primary method used for audio-visual pattern recognition.

The term \emph{Sensitive content} is often used as a reference to any media that contains content such as nudity, intense sexuality, violence, gore, and any other potentially disturbing or offensive subject.
On the other hand, a content is labeled as \emph{Safe} when that content is suitable for the general public.


% \begin{figure}[!ht]
%   \centering
%   \begin{subfigure}[b]{0.45\textwidth}
%     \centering
%     \includegraphics[width=0.8\textwidth]{img/safe.png}
%     \subcaption{Safe videos.}
%     \label{fig:samples_safe}
%   \end{subfigure}
%   \hspace{2em}
%   \begin{subfigure}[b]{0.45\textwidth}
%     \centering
%     \includegraphics[width=0.8\textwidth]{img/sensitive.png}
%     \subcaption{Sensitive videos }
%     \label{fig:samples_not_save}
%   \end{subfigure}
%   \caption{Examples of Safe and Sensitive content.}
%   \label{fig:samples}
% \end{figure}
\begin{figure*}[!ht]
    \centering
    \includegraphics[width=0.95\textwidth]{img/safe-sensitive-horizontal.png}
    \caption{Examples of safe (top row) and sensitive videos (bottom row).}
    \label{fig:samples}
    \vspace{-1em}
\end{figure*}

Figure \ref{fig:samples} illustrates these two categories.
There are four scenes with safe content on the top row, and four scenes with sensitive content on the bottom row.

%As anyone can easily access any content on the Internet, whether exploring on search engines or through social networks, some groups of people (especially children) are very vulnerable to the exposure of content not suitable for their ages. 
%This situation calls for some media control strategy managed by parents or tutors, ensuring the least exposure to sensitive content.

%Controlling the type of content uploaded to video storage services requires an automatic analysis in an accurate and efficient way. 

%In this work, we created a CNN based model for video feature extraction and validate these video features experimenting with different baseline models to detect sensitive content.
%Then we evaluate the best model in a dataset created with videos sampled from the Brazilian RNP (National Research Network) repository video@RNP.\footnote{\url{https://www.video.rnp.br}}
%In our experimentation, the best model achieves a recall of 94.4\% and an F1-score of 95.6\% for pornography class.

Other works, such as \cite{moreira2019multimodal}, share our motivations and objectives, as described in Section \ref{sec:related}. However, most of them do not use both audio and image for classification. Some use hand-crafted feature extraction methods instead of more recent CNNs that has been showing great potential in video recognition and classification.

Our work uses two CNNs: one to extract image sequence features and the other to extract audio features.
As we get one feature vector for each second of the video, we can approach the feature classification task as a time series classification, using a Recurrent Neural Network (RNN) as baseline. We also can combine those features to create a single feature vector for the entire video, which then is used as the input for other baseline classifiers.
%Our method uses a rather simpler approach for video classification and yet it still yields results significantly close to related works.


%\todo{Also talk about scene localization, scene detection}

%What is the difference between sensitive content detection and classification?

Although the term \textit{classification} can be used to define the task we are addressing, in this work we favor the term \textit{detection} over \textit{classification} to avoid leaving an open interpretation, as the term \textit{classification} is often used to include tasks with multiple classes, while our task relies specifically on binary classification. Furthermore, \textit{Detection} in this context should not be interpreted as the task to find (either time-wise or space-wise) voice content in the video.
We use detection in a more specific sense: the act of finding out if sensitive media is or is not present in the content. 

%investigar a classifciacao e conteudo sensivel em video, principalmente em videos pornograficos e vi9lencia extrema

%objetivos espcificos
%cricao de dataset
%implemwentacao de baselines
%testar a eficiencia da extracao de features ja usada no yt8m
% testar a multimodalidade
% testar abordagem sequencial e não sequencial



%colocar exelpos do que é gore e oq n é 

%Obrigado pela recomendação professora! Foram excelentes leituras e tentarei trazer alguns desses conceitos para o meu trabalho.
% Infelizmente acho que nenhum dos datasets, apesar de numerosos e variados, é compativel com a nossa definição,

% O dataset que mais se aproximou com a nossa definição foi o MediaEval 2015, porém nosso dataset vem de principalmente de gravações amadoras ou CCTV, não de cenas cinematográficas.
% Seria interessante se testar se o treino em cenas amadoras se traduzem em bons resultados em cenas de estúdio, porém para fazer isso teriamos que filtrar alguns vídeos desse dataset que não entram em nossa definição.

% As nossas definições de violência são os vídeos de extremamente violentos (gore), que incluem pelo menos um desses: Sangue, Multilação e Morte.
% Não fazem parte da nossa definição: presença de armas, lutas, discussões, acidentes de carro (que não contenham nenhum dos topicos acima), violência emocional/mental e violência animada/cartunizada.

%In this work our goal is to create and validate a approach for sensitive content detection in video.

% Some of the questions we aim to answer with this work are:
% \begin{enumerate}
%     \item How does this approach compares with the related work?
%     \item What is the impact of also using audio in the model's performance?
%     \item Can the same model have a performance higher than 90\% on both pornography and violence detection tasks?
% \end{enumerate}

%To find the answers to these questions and fulfil our goal, 
The main contributions of this work are:
\begin{enumerate}
    % Criamos um dataset para a tarefa de classificacao de conteudo sensivel, o maior do mundo até onde sabemos
    \item To our knowledge, the largest sensistive content detection dataset.
    \item To our knowledge, we have obtained the best results in this task using only a generalistic feature extraction method and generic classifiers.
    % Testamos baselines nessa tarefa para validar o dataset e o funcionamento da extração de features
    \item We trained and tested baseline classifiers on the features extracted from our dataset in order to validate both the dataset and the efficiency nad operation of the genralistic feature extraction networks.  
    % Experimentamos com diferentes configurações de classificadores, inculsive sequenciais e naõ sequenciais
    \item We compared sequential and non sequential classifiers in this task.
    % Testamos a importancia dos features de imagem e de audio separadamente
    \item We tested the importance of image and audio features in this task.
    \item We also validate our approach by testing our best baseline in a well known pornography detection dataset, 2k-pornography. 
\end{enumerate}
Although the largest dataset for this task by our knowledge, our dataset is not manually labeled, which begs the question if it is noise-less enough for any training and evaluation in this task.  
Our intent is not to replace the 2k-pornography dataset, but to be a complement it, it still is the gold standard for pornography detection, in our dataset the videos were not manually labeled by a human, so we need to validate this dataset.

To perform this task, we created a large scale dataset, extracted features from this dataset using an generalist and well known feature extraction for video classification method, and performed experiments such as compare baseline classification models, compare which type of classification model (sequential or not) performs best, and compared the importance of audio and image features. further detailed in Section \ref{sec:approach}.

This dissertation proposal is organized as follows:
In Section \ref{sec:related} we discuss some of the related work.
Then, in Section \ref{sec:approach}, we detail the proposed method to detect sensitive content in videos.
We present our dataset and metrics in Sections \ref{sec:dataset} and \ref{sec:metrics}, respectively.
Then, we propose experiments to evaluate our models in Section \ref{sec:experiments}.
Finally, in Sections \ref{sec:contrib},~\ref{sec:expdcontrib} and \ref{sec:schedule} we present, respectively, our currently already achieved contributions, our still expected contributions, and our schedule.
% %%% -*- coding: utf-8 -*-
\newpage

\chapter{Related Work}
\label{chap:related}

 %O dataset que mais se aproximou com a nossa definição foi o MediaEval 2015, porém nosso dataset vem de principalmente de gravações amadoras ou CCTV, não de cenas cinematográficas.
 
% commented out because wqe have a more recent paper from theese same authors
%Song and Kin~\cite{song2018pornographic} create a scheme for detecting pornography videos using multimodal features: Image descriptor features of the frame sequence, extracted using the VGG-16 CNN\cite{vgg}, motion features extracted using optical flow\cite{opticalflow}, and audio features extracted using a Mel-scaled spectrogram.
%The final features for each model are obtained by an average pooling of each of the features by a sample in the video.
%Each of those kinds of features is used in a single SVM classifier per type of feature, resulting in an image sequence-based detector, a motion-based detector, and an audio-based detector.
%The final decision-making is done by model stacking all detectors.
%The authors used a modified dataset based on the 2k-pornography dataset\cite{2kdataset} for training and testing.
%The results of their method are an average of 63.4\% with a 100\% true positive rate for porn, and an average of 23.5\% of the false-positive rate. 
%Our work also uses multimodal features, but we only use image sequence features and audio features, furthermore, we use an Inception-V3 CNN instead of the VGG-16 CNN for extracting image sequence features and use an Audio VGG CNN for extracting audio features.
%Although the authors achieve 100\% recall rate for pornography, which is the main goal of their task, their model also has a 23.5\% false-positive rate, which means that normal videos would be occasionally classified as pornography.
%Our aim is to also have a true positive rate as high as theirs, while still further reducing the false positive rate.

Castro~\cite{torres2018automatic} shows an implementation of a pornography video classifier using a convolutional neural network from Open pornography~\cite{mahadeokar2016open} and the dataset from Nude Detection in Video using Bag-of-Visual-Features~\cite{lopes2009nude} dataset.
The CNN does a logistic regression on each frame, resulting in a value from 0 to 1 at each frame.
The higher the value is, the higher is the likelihood of the frame being pornography.
The dataset used contained 90 non-pornography video segments and 89 pornography video segments extracted from 11 movies.
The final score for the video the max value from all frames of the video.
The experiment showed an accuracy of 81\%, and a F1-score, and Matthew’s correlation coefficient(MCC) for the pornography class of 0.8047 and 0.6343, respectively.
Although the work also approaches pornography content detection in videos problem with CNN like ours, it does not make use of audio features.
The method is also different, it performs the regression first, then it takes the max value from all frames of the video, while ours, in the non-sequential approach, combines features from all frames of the video into a single vector of features (mainly by averaging) and then performs classification on the resulting features.

% Tem o trabalho do professor novo também, ele usa convnets pra extrair as features e uma rnn pra classificar
%https://www.sciencedirect.com/science/article/pii/S0925231217312493
Wehrmann \textit{et al.}~\cite{wehrmann2018adult} classify adult content trained on the NPDI pornography video dataset ~\cite{avila2013pooling}, which consists of 802 videos, totaling 80 hours of videos, half of them with adult content.
Those videos were processed by keyframes, varying between 1 and 320 frames per video.
The selected keyframes of each video were chosen by a scene segmentation algorithm, resulting in 16727 images. % this was done by the dataset authors, not the work authors
Their architecture consists of a Convolutional Network and an Long-Short Term Memory Network (LSTM)~\cite{hochreiter1997long}.
Those models were chosen for feature extraction with CNN and sequence learning with LSTM, taking into consideration modifications on the images such as scaling and distorting.
Using this approach the authors achieved a score of 95.6\% ± 1 accuracy and 0.990 AU(ROC).
In our model, we also approached the video analysis using frame by frame processing, but but we also processed the extracted sound from each frame.% using a pre-trained Convolutional Neural Network%, instead of an untrained one.
%\hl{Resulting in an accuracy of 98\% and 97.97\% F1-score for pornography class.}

Sing \textit{et. al.}\cite{singh2019kidsguard} proposes a fine-grained approach for child unsafe video representation and detection. One of its main objectives is to optimize detection on sparsely present child unsafe content and it does so by using a VGG16\cite{vgg} Convolutional Neural Network (CNN) to encode each frame, at 1-second granularity, in 512 real values. 
Then an LSTM autoencoder is trained to output the sequence backward on those encoded frames. 
Once the LSTM autoencoder is trained, then a fully connected layer of neurons is used to fine-tune and classify each frame. 
The dataset used comprises 109,835 short-duration video clips extracted from four animes. 
The results for binary classification using safe and unsafe classes were 81\% recall for unsafe and 0.88 AUC(ROC) for unsafe class.
Although this work also has similar objectives as ours and also uses a CNN-based encoding method, ours uses both visual and audio features to encode a video. 
%Their work uses 1 frame per second granularity and ours has the same encoding rate. 
The main difference between both works is on the dataset: Theirs consists of small clips of only anime videos. Ours also uses other types of videos such as live-action and other animations. 
%In our dataset, the length of videos range from 6 seconds to 30 minutes.

Song \textit{et. al.}~\cite{song2020enhanced} proposed a multimodal stacking scheme for fast and accurate online detection of pornographic content.
Their work uses both visual and auditory features as input for their detection method. 
They use a VGG16 model and a bi-directional LSTM to extract visual features and a combination of a Mel-scaled spectrogram followed by multilayered dilated convolutions to extract audio features. 
Using only the visual and auditory features, a video classifier and an audio classifier are trained, respectively. 
By using both features together, one fusion classifier is also trained.
Then, these three component classifiers are combined in an ensemble scheme to reduce the false-negative errors and for faster detection. 
The proposed detection method yields a true positive rate of 95.40\% and a false negative rate of 4.60\% on the pornography class, totaling a recall for the pornography class of 95.40\% and accuracy of 92.33\%. 
The dataset used was the NPDI 2k-pornography\cite{2kdataset} dataset plus examples of videos with only pornographic or non-pornographic audio collected by the authors. 
This work is similar to ours because it also uses a multimodal approach to detection, albeit ours is not for pornography detection only.
It also uses the same sampling rate of a frame for each second and uses a deep learning method for extracting high-level features, which are then classified by one or more machine learning models. 
% Our work has a different dataset, comprised of pornography, gore, and violent videos for the inappropriate class and miscellaneous and educational YouTube videos as an appropriate class. 
We also use different feature extraction methods for image and audio features. 
Finally, in contrast with their ensemble approach, we use a single model to classify the extracted features from our dataset.


Moreira \textit{et.al.}~\cite{moreira2019multimodal} has similar detection focuses as ours: Pornography and Violence. 
Their method uses four multi-modal classifiers, two for audio and two for image, those classifiers were fed features from multiple handcrafted feature extraction methods. Their work is geared towards mobile device applications and also allows for sensitive scene localization.
The authors propose a method for sensitive scene localization which uses the output of four multi-modal classifiers on snippets of the video, then creates a fusion vector at each second of the video. 
Finally, they test different classifiers on the fusion vector for each task: detecting pornography and detecting violence. Their best result on the pornography task was 90.75\% accuracy and 93.53\% on the F2 metric. For the violent videos, they achieved 0.502 on the MAP2014 evaluation metric.
Some differences between this work and our are mainly its objectives: To detect if and at what time the sensitive video occurs. While our only objective is to detect if there is or not sensitive content in a video. Their method is geared towards mobile devices, while ours is geared towards video hosting platforms.
%Our definition of violent videos consists of extremely violent videos, while theirs also considers videos with minor violence such as fights.
Other differences stand out as the dataset and the methods used for feature extraction and classification. 
The Violent Scenes Dataset \cite{VSD2014} is comprised of violent scenes from movies, while ours contains real violent scenes.
We use an authorial dataset and investigate what results a deep learning-based approach to this problem can yield.


%THEIR METHOD EVALUATES ON EACH FRAME, OUR ON THE ENTIRE VIDEO}

Wang \textit{et. al.} ~\cite{wang2019porn} proposes a pornography method for use in live streams, focusing on real-time processing, their work uses multimodal features, namely, image, audio, and optical flow~\cite{horn1981opticalflow}. 
An Xception~\cite{chollet2017xception} model is used to extract spatial features from keyframes. 
To get the optical flow frames, they also use a CNN to extract the optical flow from the video, then, use another Xception model to extract the high-level optical flow features. 
Finally, they use a short-time Fourier transformation to create spectrograms and feed those spectrograms to a third Xception model and thus acquiring the extracted audio features.  
Each of the multimodal features extracted then is passed onto bidirectional GRUs\cite{dey2017GRU}, to obtain temporal context, then, to create a better-unified representation, all the features go through three interconnected Attention-gated layers, each with three Attention-gated units proposed in the paper. After obtaining the dense representation of the input types, it is applied a fully connected layer of neurons with \textit{softmax} function. Their work archives 76.33\% accuracy and runs at 66.1 fps.
In our work, we strive for detecting both violence and pornography, we use only two types of input data, image, and audio, and we use a specific CNN for each type of data, while their work focused only on detecting pornography and used the same CNN model for all three types of input. 
%We investigate whether bigger and more specialized models can create better high-level features and further increase the quality of classification.

Liu \textit{et. al.} ~\cite{liu2020analyzing} proposes a multi-modal approach to pornography detection, it uses audio-frames and visual-frames to create handcrafted low-level features based on, respectively, periodic patterns and salient regions. Once those features are extracted, they use k-means clustering to create audio and visual codebooks. 
Then, low-level audio and visual features of test videos are converted into mid-level semantic histograms via de audio or visual codebook. 
Finally, the histograms are concatenated to represent the video and a periodicity-based video decision algorithm is used to fuse the classification results of multi-modal codebooks and the results of an SVM trained on the concatenated mid-level semantic features train set.
The true positive rate of their approach achieves 96.7\% while the false positive rate is about 10\%.
%Liu \textit{et. al.} detects pornography, they do not detect violence, and also use handcrafted features such as Region Of Interest (ROI) extraction and skin-color segmentation.
%Whereas our approach uses a fully automatic feature extraction method based on CNNs and our feature fusion method consists of just a concatenation followed by a classifier.

Most related works focus on pornography detection alone, while ours aims at detecting either pornographic or violent content. Moreover, some of them only use image-frame features, whereas we use both audio and image-frame features. We also use deep learning feature extraction methods instead of hand-crafted ones. Feature extraction method, classification method and dataset of each related work are available in Table \ref{tab:related}.
Finally, a central difference is our dataset: Ours contains violent scenes and is significantly larger than most datasets used on other related works.

\begin{landscape}
\begin{table}[]
\scriptsize
\centering
\caption{Related work comparative table.}
\begin{tabular}{l|l|l|l|l|l|l|l|l|l}
Paper                                & Task                                                                    & Feature extraction method                                                                                                                                           & Classifier                                                                                 & F2-Score                                                        & Recall  & Accuracy                                                                                 & F1-Score & AUC (ROC) & Dataset                                                                                                                             \\ \hline
Castro \cite{torres2018automatic}           & \begin{tabular}[c]{@{}l@{}}Pornography\\ detection\end{tabular}         & Resnet 50 for image.                                                                                                                                                & Resnet-50                                                                                  & 0,7798                                                          & 0,7640  & 0,8160                                                                                   & 0,8047   & NA        & \begin{tabular}[c]{@{}l@{}}Open pornography\\  + Nude Detection in \\ Video using \\ Bag-of-Visual-Features \\ dataset\end{tabular} \\ \hline
Wehrmann et al. \cite{wehrmann2018adult}    & \begin{tabular}[c]{@{}l@{}}Pornography\\ detection\end{tabular}         & ResNet-101 for image.                                                                                                                                               & LSTM                                                                                       & 0,9520                                                          & 0,9501  & 0,9560                                                                                   & 0,9548   & 0,9900    & \begin{tabular}[c]{@{}l@{}}NPDI pornography\\ video dataset\end{tabular}                                                            \\ \hline
Sing et. al. \cite{singh2019kidsguard}      & \begin{tabular}[c]{@{}l@{}}Sensitive\\ content\\ detection\end{tabular} & VGG16 for image.                                                                                                                                                    & \begin{tabular}[c]{@{}l@{}}LSTM \\ + FC\end{tabular}                                       & NA                                                              & 0,8100  & NA                                                                                       & NA       & 0,8800    & Author (Animes)                                                                                                                     \\ \hline
Song et. al. \cite{song2020enhanced}        & \begin{tabular}[c]{@{}l@{}}Pornography\\ detection\end{tabular}         & \begin{tabular}[c]{@{}l@{}}VGG16\\ + BiLSTM for image;\\ Mel-scaled spectrogram\\ + Multilayered dilated\\ convolutions for audio.\end{tabular}                     & \begin{tabular}[c]{@{}l@{}}Early \\ + Late fusion FC\\ voting\end{tabular}                 & NA                                                              & 95.40\% & 0,9233                                                                                   & NA       & NA        & \begin{tabular}[c]{@{}l@{}}NPDI\\ 2k-pornography\end{tabular}                                                                       \\ \hline
Moreira et.al. \cite{moreira2019multimodal} & \begin{tabular}[c]{@{}l@{}}Sensitive\\ content\\ detection\end{tabular} & \begin{tabular}[c]{@{}l@{}}HOG \\ for image;\\ TRoF \\ for space-temporal description;\\ MFCC and \\ prosodic features\\  for audio.\end{tabular}                   & \begin{tabular}[c]{@{}l@{}}Thresholding \\ (Pornography);\\ SVM\\ (Violence).\end{tabular} & \begin{tabular}[c]{@{}l@{}}93,53\%\\ (Pornography)\end{tabular} & NA      & \begin{tabular}[c]{@{}l@{}}90,75\%\\(Pornography);\\0.502 MAP2014\\(Violence).\end{tabular} & NA       & NA        & \begin{tabular}[c]{@{}l@{}}NPDI\\ 2k-pornography +\\ Violent Scenes Dataset\\ (MediaEval 2014)\end{tabular}                         \\ \hline
Wang et al. \cite{wang2019porn}             & \begin{tabular}[c]{@{}l@{}}Pornography\\ detection\end{tabular}         & \begin{tabular}[c]{@{}l@{}}Xception for image;\\ Optical flow \\ + Xception for motion;\\ Short-time Fourier\\ transformation \\ + Xception for audio.\end{tabular} & \begin{tabular}[c]{@{}l@{}}bidirectional GRUs \\ + Attention \\ + FC\end{tabular}          & NA                                                              & NA      & 76,33\%                                                                                  & NA       & NA        & \begin{tabular}[c]{@{}l@{}}BJUT streamer\\  dataset (Author)\end{tabular}                                                           \\ \hline
Liu et. al. \cite{liu2020analyzing}         & \begin{tabular}[c]{@{}l@{}}Pornography\\ detection\end{tabular}         & \begin{tabular}[c]{@{}l@{}}Skin color detection \\ + Face detection \\ + Salient regions detection \\ + SURF for image.\end{tabular}                                & SVM                                                                                        & NA                                                              & NA      & 96,70\%                                                                                  & NA       & NA        & Author                                                                                                                             
\end{tabular}
\label{tab:related}
\end{table}
\end{landscape}
% %%% -*- coding: utf-8 -*-
\newpage

\chapter{Theory and technical background}
\label{chap:theory}
% \section{Method and experimentation}
\label{sec:method}
In this section we detail our method for sensitive content detection in video. We split our approach into three parts: feature extraction, feature fusion and feature classification, as illustrated in Figure \ref{fig:model}.
\begin{figure*}[!ht]
    \centering
    \includegraphics[width=0.9\textwidth]{img/model-2.png}
    \caption{Bimodal architecture for NSFW video classification.}
    \label{fig:model}
    \vspace{-1em}
\end{figure*}

In the feature extraction stage, firstly we split the frames and audio from the video; then, for each media, we use a CNN to extract the features (or embeddings) from each simultaneous video segment. In the second stage, Feature Fusion, we concatenate both audio and frame features. If the classification model is not sequential, we also aggregate the features in this stage. Finally, in the feature classification stage we feed one of the classification models to be experimented with.

\subsection{Video Embeddings Extraction}
\label{subsec:video_features}

CNNs tend to learn low-level features (\textit{e.g.}, in the visual domain: edges, corners, contours) at their first layers. At the intermediate and final layers, the combination of these features helps to extract more complex features, resulting in a vector of continuous values, referred to as \textit{embeddings}, that might be used for classification and other tasks. In this work, we use two benchmark CNNs to extract both image and audio \textit{embeddings} by using a transfer learning technique~\cite{tan2018survey}.



%Afterwards, we apply PCA (+ whitening) to reduce feature dimensions to 1024, followed by quantization (1 byte per coefficient).
%These two compression techniques reduce the size of the data by a factor of 8. The mean vector and covariance matrix for PCA was computed on all frames from the Train partition. We quantize each 32-bit float into 256 distinct values (8 bits) using optimally computed (non-uniform) quantization bin boundaries. We confirmed that the size reduction does not significantly hurt the evaluation metrics. In fact, training all baselines on the full-size data (8 times larger than what we publish), increases all evaluation metrics by less than 1%.

\todo{explicar pq escolheu as cnns (youtube) e qual foram os paramretros usados (pca e hiperparametros)}
By using the feature extraction method created for the Youtube-8m benchmark, we can test an feature extraction method that is powerful enough to represent features that can be in multiple tasks, such as multi-label video classification, video recommendation, and human activity recognition.

"Since the video-level representations are unsupervised (extracted independently of the labels), these representations are far less specialized to the labels associated with the current dataset, and can generalize better to new tasks or video domains."~\cite{abu2016youtube}

In order to validate our dataset, we used the same feature extraction method used in the Youtube-8m dataset challenges~\cite{abu2016youtube}, both networks were pre-trained and frozen and they were not retrained for application sensitive content classification. Which gives future works an opportunity to develop even more efficient and smaller feature extraction networks for this specific task.


To generate image frame features and audio features we decode each video at approximately 1 frame-per-second and feed an InceptionV3  network~\cite{szegedy2016rethinking} pre-trained on the ImageNet\footnote{\url{http://www.image-net.org/}} dataset.
 
We also make use of a AudioVGG~\cite{hershey2017cnn} network with pre-trained weights in the Audioset\footnote{\url{https://research.google.com/audioset/}} dataset to extract the audio embeddings. Each of these CNNs were used as published by their authors; the only modification was the removal of classification layers in both CNNs to obtain their respective embeddings.

Next, we apply Principal Component Analysis (PCA)~\cite{wold1987principal} to each of the outputs to reduce the dimensions of both embeddings and to generate feature vectors of size 1024 and 128 for frame and audio embeddings respectively.
%DETALHAR MAIS AQUI DE QTO PRA QTO O PCA REDUZ

\subsection{Feature Fusion}
\label{subsec:feature_fusion}

Once we have the features from both image and audio, we should make a decision about which method is best to fuse the information from these different domains. Snoek et al. \cite{snoek2005featurefusion} presents two main strategies for information fusion in semantic video analysis: \emph{Early fusion} methods, which works directly with the extracted features, and \emph{Late fusion} methods, which operates on classification outputs from specialized models.
%\todo{For a more recent survey about data fusion and multimedia retrieval, please refer to \cite{jiang2013features}.}
In this work, because we have high abstraction level features, we opted to investigate the most simple approach, which is to use a single model on the concatenated features from both media inputs.

Specifically, we concatenate both image and audio embeddings extracted in the current frame and audio window in order to compose the final embeddings as a sequence of the same size of the number of seconds of the video. After this concatenation, each time-step has 1,152 features: 128 audio features and 1024 frame features.

Notice that with this approach, the video is transformed into a time series, and to use it in non-sequential models (\textit{e.g.}~SVM, KNN, and MLP) we need to turn this sequence into a single feature vector that represents the whole video. In our setting, we did that by taking the average, median, standard deviation, min, and max values for each feature to represent the entire video. In summary, we turn the sequence of features with size $n$ and shape $n$ by 1,152 into a single feature with shape 1 by 5,760.

\subsection{Classifiers}
\label{subsec:classifiers}

% \todo[inline]{porque foi escolhidos esses modelos de classficação, qual a relação deles com o nosso tipo de dado de vídeo.}

For the feature classification task, we will investigate both sequential models (which use the extracted embeddings in a time series format), and non-sequential ones (which use a single aggregated embeddings vector).
We want to experiment with both approaches in order to investigate if a more compact format, such as the single embeddings vector, can yield results at least as good (or even better) than the full feature sequence data.
%Furthermore, we can test if a sequential model can outperform a non-sequential model in specific cases that demand long term memory, such as long videos with very small sensitive scenes.
As an example, one can think of a long video that has a pornographic scene in one second out of its entirety. 
In a non-sequential representation of the extracted features, this short pornographic fragment could be left ``hidden'' among the other non-pornographic frames of the video, as illustrated in Figure \ref{fig:model-non-sequence}.
\begin{figure*}[!ht]
    \centering
    \includegraphics[width=0.9\textwidth]{img/model-non-sequence.png}
    \caption{Sequential features with aggregation, the sensitive scene (red) might vanish among the the other scenes during aggregation.}
    \label{fig:model-non-sequence}
    \vspace{-1em}
\end{figure*}
In a sequential representation, although time series classifiers usually output a prediction after reading the entire sequence, the embedding vectors of each second of the video would not be aggregated and thus could be analysed section by section, as illustrated in Figure \ref{fig:model-sequence}.

\begin{figure*}[!ht]
    \centering
    \includegraphics[width=0.9\textwidth]{img/model-sequence.png}
    \caption{Sequential features with no aggregation. In a output after reading the entire sequence, this can also be susceptible to information vanishing.}
    \label{fig:model-sequence}
    \vspace{-1em}
\end{figure*}
Although a sequential representation contains possibly much more redundant data than the non-sequential one, it could give the sequential classification model an important edge of detail over the less granular non-sequential ones.
% To do the feature classification task, we experimented four well known classification mohave dels, one using extracted video features in the time series format, and three that use a single aggregated feature vector.

For the sequential classification model, we chose the Long Short-Term Memory (LSTM)\cite{hochreiter1997long} networks.
It has been a commonly used time series classification baseline model. %\todo{mayber say we'll test GRUs?}

For the non-sequence models, we chose Support Vector Machines (SVM)~\cite{cortes1995support}, K-Nearest Neighbors (KNN)~\cite{peterson2009k}, and Multilayer Perceptron (MLP)~\cite{haykin2009neural}.
Among all of the experimented models, the \textit{Support Vector Machine (SVM)} is the most used in the literature.
% \begin{enumerate}[leftmargin=*]
% \item 
It is a classification model in which the data is mapped into a higher dimension input space, where an optimal separating hyper-plane is constructed.
% These decision surfaces are found by solving a linearly constrained quadratic programming problem.
% \item 
The second model, \textit{K-Nearest Neighbors (KNN)} uses distance measure between training samples so that the k-nearest neighbors always belong to the same class, while samples from different classes are separated by a large margin. 
It was chosen because it used also by related work, although it is a simple classification method.
% \item 
The third model is the \textit{Multilayer-perceptron (MLP)}, which contains layers of nodes: an input layer, an output layer and various hidden layers in between. 
This one was selected because it is also commonly used as a final classifier on deep neural networks.  
% The number of layers used is problem dependent, as is the number of nodes in each hidden layer.
% The weights are adjusted by local optimization using a set of feature vectors so that the network produces the optimal expected output.
% \item 
%Lastly, \textit{Long short-term memory (LSTM)}, different than the feed-forward neural networks, process the entire sequences of data using feedback connections.
% \end{enumerate}

In next the Section, we present the \textit{dataset} created to train and evaluate these classification models.

\section{Dataset}\label{sec:dataset}
% Falar do sampling inicial, da presentatividade dos sets
% falar do balanceamento e do sampling do conjunto de teste
% Falar das talbeas no texto
% colocar o resto das tabelas talvez no appendix
% estatisticas depois de dropar os maiores e os menores videos, sem balanceamento
% usamos o balanceamento dropando apesa porn videos pq os gore sao poucos
% iremos distribuir o dataset na versao balanceada e sem balanceamento
% Comparar nossas escolhas de corte de tempo com o do yt 
% Checked for duplicates by name and size and by fdupes, sem garantias de de não ter subvideos


% Before clipping video based on duration
% Mean:  0:06:18.929511  STD:  0:12:18.237458
% There are 1355 (1.05%) videos longer than 00:30:56
% There are 116 (0.09%) videos shorter than 00:00:05

% Videos:
% Total dataset: 127075
% Total size of videos (calculated individually): 3.5TiB
% Total duration of videos (calculated indiviadually): 11806:21:13
% Improper videos: 67424
% Total size of videos (calculated individually): 1.2TiB
% Total duration of videos (calculated indiviadually): 6953:27:41
% Proper videos: 59651
% Total size of videos (calculated individually): 2.2TiB
% Total duration of videos (calculated indiviadually): 4852:53:31
% Total size of features: '896.3GiB'

% Hot much detail do we go on about the dataset's subclasses? (subclasses por porn and yt, no subclasses on gore)

Our \textit{dataset} is structured into two main classes: videos containing ``safe'' content and videos containing sensitive content.
It is divided into 59,651 safe videos and 67,424 videos with sensitive content.

\begin{table}
\centering
\label{tab:general-stats}
\caption{General statistics of the two main classes of the dataset, tag coverage refers to main tag annotation existence (videos may also have sub tags but no main tag).}
\begin{tabular}{c|r|r} 
\multicolumn{1}{l|}{} & Sensitive  & Safe        \\ 
\hline
Video Count           & 67424      & 59651       \\ 
\hline
Total Duration        & 6953:27:41 & 4852:53:31  \\ 
\hline
Mean Duration         & 00:06:11   & 00:04:52    \\ 
\hline
STD Duration          & 00:04:12   & 00:03:26    \\ 
\hline
Max Duration          & 00:30:55   & 00:30:55    \\ 
\hline
Min Duration          & 00:00:05   & 00:00:05    \\ 
\hline
Total Size            & 1.2TiB     & 2.2TiB      \\ 
\hline
Mean Size             & 19.3MiB    & 39.0MiB     \\ 
\hline
STD Size              & 35.4MiB    & 42.3MiB     \\ 
\hline
Features Size         & 519.4GiB   & 376.8GiB    \\ 
\hline
Tag coverage          & 65392      & 51011       \\ 
\hline
Tag coverage (\%)     & 96,9862    & 85,5157     \\
\end{tabular}
\end{table}
% \subsection{SFW Videos}\label{sec:safe_videos}

For \textit{safe content}, we extracted 50,988 videos from Youtube8M\footnote{\url{https://research.google.com/youtube8m}}~\cite{abu2016youtube}. We chose this dataset because of the size of the dataset (8 million videos) and because of the wide variety of video classification challenges it supports.

We also included 8,663 videos collected from Youtube, hereby refered as cherry-picked, those videos were selected for the purpose of incremeting the amount of "hard" videos, which are videos that could possibly be misclassified as sensitive, such as MMA, breastfeeding, pool party, beach and other videos that have a higher amount of skin exposure. The amount of cherry-picked videos collected are listed by its query in Table \ref{tab:non-yt-count}. The collection was made by automated means, an script automatically searches and downloads all videos from the first 100 result pages.
% One of the objectives of the project is to evaluate the presence of inappropriate videos in educational video repositories. Therefore, in selecting your own videos within YouTube8M, we have chosen videos from \ "Jobs \& Education" and "News" top-categories.
% We made this choice because these videos generally have the "talking head" format. We believe that this format is common in an educational video repository such as video@RNP.
% \footnote{\url{https://video.rnp.br}}.

% We also included the Cholec80 \cite{twinanda2016endonet} dataset, it contains 80 videos of cholecystectomy surgeries performed by 13 surgeons. All videos from the Cholec80 dataset were labeled as safe, since videos of surgery are usual in an educational context.

% \subsection{NSFW Videos}\label{sec:nsfw_videos}

For \textit{sensitive content}, we collected pornography and violent videography (thereafter referred to as \textit{gore}) from websites. For the porn category, we collected 54,549 videos from XVideos\footnote{\url{https://info.xvideos.com/db}}. We chose this source because of the database size (7 million videos) and because of the amount and variety of annotations. In this database, each video has one main tag, totalling 70 main tags, and none or many subtags (user-created). To select the videos in this database, we distributed videos in these tags to maintain a proportion equal to the original XVideos database. In particular, to prevent lower-quantity tags from disappearing, we have defined a minimum of 10 pornographic videos for each tag. 

For the gore content, we used a web crawler to extract 2,356 gore videos from various websites dedicated to gore media, such as, BestGore\footnote{\url{https://www.bestgore.com/}} and GoreBrasil\footnote{\url{https://www.gorebrasil.com}}. 

%\todo{o set de teste de gore n tem gore de verdade, incluir?}
%For testing purposes we collected a separate group of 671 violent videos from youtube channel BustedLocals\footnote{\url{https://www.youtube.com/channel/UCpKimtjklYBgSKfSJrnXLRA}}
\begin{table}
\centering
\label{tab:granular-stats}
\caption{Granular statistics of the dataset: Videos collected from Youtube, pornographic videos, and gore videos.}
\begin{tabular}{c|r|r|r} 
\multicolumn{1}{l|}{} & \multicolumn{1}{c|}{Porn} & \multicolumn{1}{c|}{Gore} & \multicolumn{1}{c}{YouTube}  \\ 
\hline
Video Count           & 65068                     & 2356                      & 59651                         \\ 
\hline
Total Duration        & 6900:17:38                & 53:10:02                  & 4852:53:31                    \\ 
\hline
Mean Duration         & 00:06:21                  & 00:01:21                  & 00:04:52                      \\ 
\hline
STD Duration          & 00:04:10                  & 00:01:26                  & 00:03:26                      \\ 
\hline
Max Duration          & 00:30:55                  & 00:16:56                  & 00:30:55                      \\ 
\hline
Min Duration          & 00:00:05                  & 00:00:05                  & 00:00:05                      \\ 
\hline
Total Size            & 1.2TiB                    & 15.8GiB                   & 2.2TiB                        \\ 
\hline
Mean Size             & 19.8MiB                   & 6.9MiB                    & 39.0MiB                       \\ 
\hline
STD Size              & 35.9MiB                   & 13.9MiB                   & 42.3MiB                       \\ 
\hline
Features Size         & 515.3GiB                  & 4.1GiB                    & 376.8GiB                      \\ 
\hline
Tag coverage          & 63036                     & 2356                      & 51011                         \\ 
\hline
Tag coverage (\%)     & 96,8771                   & 100,0000                  & 85,5157                       \\
\end{tabular}
\end{table}

We hold out our dataset for testing and validating our approach: 10\% of the safe videos, then 10\% of gore videos, and sample a number of pornography to to match the amount of safe videos minus the amount of gore test samples, so that the test subset has a balanced amount of sensitive and safe videos, while keeping a valid amount of gore videos. 

\begin{table}
\centering
\label{tab:subset-stats}
\caption{Test subset statistics.}
\begin{tabular}{c|r|r|r} 
\multicolumn{1}{l|}{} & \multicolumn{1}{c|}{Safe} & \multicolumn{1}{c|}{Pornography} & \multicolumn{1}{c}{Gore}  \\ 
\hline
Video Count           & 5968                      & 5732                             & 236                        \\ 
\hline
Total Duration        & 443:35:36                 & 574:03:29                        & 05:22:20                   \\ 
\hline
Mean Duration         & 00:04:27                  & 00:06:00                         & 00:01:21                   \\ 
\hline
STD Duration          & 00:02:32                  & 00:03:44                         & 00:01:48                   \\ 
\hline
Max Duration          & 00:29:01                  & 00:30:29                         & 00:16:56                   \\ 
\hline
Min Duration          & 00:00:07                  & 00:00:05                         & 00:00:07                   \\ 
\hline
Total Size            & 204.8GiB                  & 80.2GiB                          & 1.5GiB                     \\ 
\hline
Mean Size             & 35.1MiB                   & 14.3MiB                          & 6.6MiB                     \\ 
\hline
STD Size              & 35.0MiB                   & 25.6MiB                          & 14.4MiB                    \\ 
\hline
Features Size         & 34.5GiB                   & 42.6GiB                          & 424.1MiB                   \\ 
\hline
Tag coverage          & 5679                      & 5695                             & 236                        \\ 
\hline
Tag coverage (\%)     & 951.575                   & 993.545                          & 100                        \\
\end{tabular}
\end{table}


As a complementary test dataset, we selected the NPDI 2k-pornography dataset~\cite{avila2013pooling}, it contains 1000 non-pornographic videos and 1000 pornographic videos. Those non-pornographic videos are comprised of ``hard'' and ``easy'' videos according to the likelihood of misclassification. Some examples of ``hard'' videos are those with high amounts of exposed skin, such as swimming and sumo fighting videos.

\begin{table}
\centering
\label{tab:2kdataset-stats}
\caption{NPDI 2k-pornography dataset statistics.}
\begin{tabular}{c|r|r} 
\multicolumn{1}{l|}{} & \multicolumn{1}{c|}{Porn} & \multicolumn{1}{c}{Non-Porn}  \\ 
\hline
Video Count           & 1000                      & 1000                           \\ 
\hline
Total Duration        & 100:30:32                 & 40:26:06                       \\ 
\hline
Mean Duration         & 00:06:01                  & 00:02:25                       \\ 
\hline
STD Duration          & 00:05:49                  & 00:02:17                       \\ 
\hline
Max Duration          & 00:33:40                  & 00:20:16                       \\ 
\hline
Min Duration          & 00:00:05                  & 00:00:02                       \\ 
\hline
Total Size            & 26.4GiB                   & 18.5GiB                        \\ 
\hline
Mean Size             & 27.0MiB                   & 18.9MiB                        \\ 
\hline
STD Size              & 31.1MiB                   & 21.9MiB                        \\ 
\hline
Features Size         & 7.6GiB                    & 3.1GiB                         \\ 
\hline
Tag coverage          & 1000                      & 1000                           \\ 
\hline
Tag coverage (\%)     & 100                       & 100                            \\
\end{tabular}
\end{table}

\section{Metrics}\label{sec:metrics}

% in binary classification, recall of the positive class is also known as “sensitivity”; recall of the negative class is “specificity”.
%https://en.wikipedia.org/wiki/Sensitivity_and_specificity
%https://en.wikipedia.org/wiki/Precision_and_recall

To evaluate each experiment and our approach, we will use Precision (P), Recall (R) and, most importantly, the weighted F2-score. In this section we present a contextualized explanation these metrics.

% FROM: de morereira et al 2019
%For assessing the performance of the pornography locators, we re-port thenormalized classification accuracyrate (ACC), and theF2measure(F2). Prior to explaining ACC, we need to definerecallandspecificityfrom the point of view of pornography localization.  Specificity, in turn, measures the capacity of alocator to correctly identify truly negative video seconds as so. A spe-cificity of only 50%, for example, means the system mislabels one inevery two seconds of non-pornographic content, wrongly identifying itas sensitive. In this vein, ACC is the mean of recall and specificity. Ahigher accuracy indicates a higher capability of separating porno-graphic video seconds from non-pornographic ones.F2measure, in turn, is a more complex metric that depends also onthe concept ofprecision. From the point of view of pornography loca-lization, precision expresses how many seconds are truly relevant (i.e.,pornographic), among all the ones that a locator identifies as such.Therefore, F2is the weighted harmonic mean of recall and precision,which gives twice more weight to recall than to precision, by means of a=β2parameter.Eq. (7)depicts the original Fβformula=+ ×××+Fβprecision  recallβ   precision  recall(1    ),β22(7)in which we use=β2. In doing so, F2lets us pay more attention to therecall of the solutions, rather than to their precision. This is usefulbecause, in the case of pornographyfiltering, false-negative answers areworse than the false-positive ones. It is less prejudicial to wrongly denythe access to non-pornographic content, than to wrongly disclose por-nographic content. Hence, we can consider that a solution with higherF2measure is better, because it cares more about how many porno-graphic video seconds are really beingfiltered out (recall), instead ofhow many“supposedly”positive seconds are indeed pornographic(precision)



In the context of sensitive content detection, \textit{true positives} are videos predicted as sensitive and are in fact, sensitive. Likewise, \textit{true negatives} are videos predicted as safe and are indeed safe. \textit{False positives} are videos predicted as sensitive, but were safe, the same goes for \textit{false negatives}, which are videos that were predicted as safe, but were predicted as sensitive.

% from the daniel moreira paper:
% thenormalized classification accuracyrate (ACC)ACC is the mean of recall and specificity. Ahigher accuracy indicates a higher capability of separating porno-graphic video seconds from non-pornographic ones

Precision (Equation~\ref{equation:precision}) measures how many videos predicted as sensitive (both true positives and false positives) are truly sensitive. The Recall (Equation~\ref{equation:recall})  measures how many truly positive videos were correctly identified.

\begin{multicols}{2}
  \begin{equation}
    \label{equation:precision}
    P = \frac{TP}{TP + FP}
  \end{equation}

  \begin{equation}
    \label{equation:recall}
    R = \frac{TP}{TP + FN}
  \end{equation}
  
\end{multicols}

Where $TP, TN, FP$, and $FN$ denote the examples that are true positives, true negatives, false-positives, and false negatives, respectively.

\begin{equation}
\label{equation:fbeta}
F_\beta = (1+\beta^2) \times \frac{P \times R}{(\beta^2 \times P) + R}
\end{equation}

The $F_\beta$-score, defined in Equation~\ref{equation:fbeta}, evaluates the classifier by the harmonic mean between Precision and Recall. To account for label imbalance, after calculating the F2-score metrics for each label, we find their average weighted by support (the number of true instances for each label). 

Most related works use either F1-score ($\beta=1$) or F2-score ($\beta=2$) metrics as their main evaluation metric. While the F1-score represents an balanced performance metric, the F2-score gives twice more weight to the recall than to precision, which means that the metric is more focused on the recall of a solution.

% in the case of pornographyfiltering, false-negative answers areworse than the false-positive ones. It is less prejudicial to wrongly denythe access to non-pornographic content, than to wrongly disclose por-nographic content. Hence, we can consider that a solution with higherF2measure is better, because it cares more about how many porno-graphic video seconds are really beingfiltered out (recall), instead ofhow many“supposedly”positive seconds are indeed pornographic(precision)
In this work, the F2-score represents an overall performance metric, while the precision and recall metrics can give insights on what the classifier model is doing better and what to improve. We chose the weighted F2-score as our main evaluation metric because when detecting sensitive content it is more important to predict a truly sensitive video than to predict a safe video as sensitive.

\section{Proposed Experiments}\label{sec:experiments}


In our proposed experiments, we evaluate the performances of baseline classifiers over the video \textit{embeddings} that were extracted from our dataset, described in Section \ref{sec:dataset}. Then choose the best performing classifier during validation stage and test its performance on the \textit{test sets}.  
%\todo{maybe add a image for the process?}
We designed a set of experiments that might help us find insights and assess the performance and shortcomings of our approach.  

% In this work our goal is to create and validate a approach for sensitive content detection in video.

% Some of the questions we aim to answer with this work are:
% \begin{enumerate}
%     \item How does this approach compares with the related work?
%     \item What is the impact of also using audio in the model's performance?
%     \item Can the same model have a performance higher than 90\% on both pornography and violence detection tasks?
% \end{enumerate}

Our objective with these experiments is to attest the quality of our approach at detecting sensitive content on video.
%, while also answering our research questions, stated in Section \ref{sec:introduction}.
% In this experiment, our objective is to attest to the quality of our video \textit{embeddings}.
 
% In the next subsections, we discuss the experiment setup, used metrics, and our findings. In Subsection \ref{subsec:config} we describe the training configuration for each model.
% Next, in Subsection \ref{sec:metrics} we describe the evaluation metrics.
% And finally, in Subsection \ref{subsec:results} we present our empirical findings.
\begin{enumerate}[start=0,label={(\bfseries E\arabic*):}]
\item The sensitive content detection task: This is the main experiment of this work, in this experiment we test the capabilities of our approach and the best performing classification model on our test subset.%\todo{maybe referenciar ao subset de teste do nosso dataset?}

\item The pornography detection task: In this experiment we evaluate our approach and the best performing classification model on the pornography detection task using our test subset.

\item The gore detection task: In this experiment we evaluate our approach and the best performing classification model on the gore detection task using our test subset.

\item Test with training only on image features: In this experiment we evaluate our approach on the our test subset using the audio features only.

\item Testing on the pornography-2k:  In this experiment we evaluate our approach on the pornography-2k dataset.

\item Testing pornography audio only videos: In this experiment we evaluate our approach on the pornography-2k dataset using the audio features only.

\item Testing on the gore test set: In this experiment we evaluate our approach on the gore test subset.

\item Testing gore audio videos: In this experiment we evaluate our approach on the gore test subset using the audio features only.
%\item Testing on the medical images set
%\item esting MEDIAEVAL Violent Scenes Dataset
\item Investigating misclassified videos in the test sets: In this experiment we search for insights on what videos our approach fails to correctly detect sensitive content.
% \item sequence classification models vs non-sequence models
\end{enumerate}

Based on those experiments we established the following tasks for the completion of this work:

\begin{enumerate}[start=0,label={(\bfseries T\arabic*):}]
\item Dataset Collection (100\% Complete)
\item Feature Extraction (100\% Complete)
\item Dataset Preparation (50\% Complete)
\item Model Training and Experimentarion (20\% Complete)
\end{enumerate}
% \input{body/05_Discussion.tex}
% \input{body/06_Final_Considerations.tex}

% \newpage
% % \addcontentsline{toc}{section}{Referências}
% \addcontentsline{toc}{section}{References}

% \bibliographystyle{sbc}
% % \bibliographystyle{plainnat}
% \bibliography{sbc-template}

% \input{body/07_Appendix.tex}
% \end{document}
